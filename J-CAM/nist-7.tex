\section{Benchmark NIST-7 "Boundary Line Singularity"}
\label{sec:bench-7}

This is a singularity problem with the solution that is singular along the left part of the boundary.
The equation solved in this problem is the Poisson's equation.

\begin{equation} \label{boundary-line-singularity}
-\Delta u = f
\end{equation}
in the domain $\Omega = (0, 1)^2$, equipped with Dirichlet boundary conditions
given by the exact solution. The exact solution is
$u(x,y) = x^{\alpha}$,
where $\alpha \geq 0.5$ determines the strength of the singularity.
The right-hand side $f$ is calculated by inserting exact solution into (\ref{boundary-line-singularity}).
The solution of NIST-7 with $\alpha = 0.6$ is shown in Fig. \ref{fig:sln-nist07}.

\begin{figure}[!ht]
\centering
\includegraphics[height=6cm]{nist/nist-7/solution.png}
\caption{The solution to NIST-7 benchmark problem.}
\label{fig:sln-nist07}
\end{figure}

\begin{figure}[!ht]
\centering
\includegraphics[height=5cm]{nist/nist-7/mesh_hp_aniso_init.png}\ \
\includegraphics[height=5cm]{nist/nist-7/mesh_hp_aniso.png}
\caption{Initial mesh (left) and final mesh (right) with 88 DOF and the resulting relative error estimate in $H^1$-norm of 1.46348 \% for $hp$-FEM with anisotropic refinements.}
\label{fig:nist-7-hp-aniso}
\end{figure}

\begin{figure}[!ht]
\centering
\includegraphics[height=5cm]{nist/nist-7/mesh_h1_aniso.png}\ \
\includegraphics[height=5cm]{nist/nist-7/mesh_h2_aniso.png}
\caption{Final mesh for $h$-FEM with linear and quadratic elements.}
\label{fig:nist-7-h-aniso}
\end{figure}

Figs. \ref{fig:nist-7-conv} compare all
three approaches to automatic adaptivity from the point
of view of DOF and CPU convergence.

\begin{figure}[!ht]
\centering
\includegraphics[height=5cm]{nist/nist-7/conv_dof_aniso.png}\ \
\includegraphics[height=5cm]{nist/nist-7/conv_cpu_aniso.png}
\caption{DOF and CPU time convergence graphs.}
\label{fig:nist-7-conv}
\end{figure}

