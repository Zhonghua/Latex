\section{Benchmark NIST-4 "Peak"}
\label{sec:bench-4}

The solution of this problem exhibits an exponential peak in the interior of the domain.
The equation solved in this benchmark problem is the Poisson's equation.

\begin{equation} \label{poisson-peak}
-\Delta u = f
\end{equation}
in the domain $\Omega = (0, 1)^2$, equipped with Dirichlet
boundary conditions given by the exact solution.
The exact solution is
$u(x,y) = e^{-\alpha ((x - x_{loc})^{2} + (y - y_{loc})^{2})}$,
where $(x_{loc}, y_{loc})$ is the location of the peak,
and $\alpha$ determines the strength of the peak.
The right-hand side $f$ is calculated by inserting exact solution into (\ref{poisson-peak}).
The solution of NIST-4 with $\alpha = 1000$,
$(x_{loc}, y_{loc}) = (0.5, 0.5)$ is shown in Fig. \ref{fig:sln-nist04}.

\begin{figure}[!ht]
\centering
\includegraphics[height=5cm]{nist/nist-4/solution.png}
\caption{The solution to NIST-4 benchmark problem.}
\label{fig:sln-nist04}
\end{figure}

\begin{figure}[!ht]
\centering
\includegraphics[height=3.7cm]{nist/nist-4/mesh_h1_aniso.png}
\includegraphics[height=3.7cm]{nist/nist-4/mesh_h2_aniso.png}
\includegraphics[height=3.7cm]{nist/nist-4/mesh_hp_aniso.png}
\caption{
Final mesh (left) with 58253 DOF and the resulting
relative error estimate in $H^1$-norm of 5.72234e-01 \% for $h$-FEM with linear elements.
Final mesh (middle) with 51473 DOF and the resulting
relative error estimate in $H^1$-norm of 1.39525e-02 \% for $h$-FEM with quadratic elements.
Final mesh (right) with 1561 DOF and the resulting 
relative error estimate in $H^1$-norm of 7.02865e-03 \% for $hp$-FEM with anisotropic refinements.}
\label{fig:nist-4-hp-aniso}
\end{figure}

%\begin{figure}[!ht]
%\centering
%\includegraphics[height=5cm]{nist/nist-4/mesh_h1_aniso.png}\ \
%\includegraphics[height=5cm]{nist/nist-4/mesh_h2_aniso.png}
%\caption{Final mesh for $h$-FEM with linear and quadratic elements.}
%\label{fig:nist-4-h-aniso}
%\end{figure}

Figs. \ref{fig:nist-4-conv} compare all
three approaches to automatic adaptivity from the point
of view of DOF and CPU convergence.

\begin{figure}[!ht]
\centering
\includegraphics[height=5cm]{nist/nist-4/conv_dof_aniso.png}\ \
\includegraphics[height=5cm]{nist/nist-4/conv_cpu_aniso.png}
\caption{DOF and CPU time convergence graphs.}
\label{fig:nist-4-conv}
\end{figure}

