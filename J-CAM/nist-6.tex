\section{Benchmark NIST-6 "Boundary Layer"}
\label{sec:bench-6}

Solution of this problem has a boundary layer along the right and top sides of the domain.
It is a convection-diffusion equation with first order terms.

\begin{equation} \label{boundary-layer}
-\epsilon \nabla^{2} u + 2\frac{\partial u}{\partial x} + \frac{\partial u}{\partial y}= f
\end{equation}
in the domain $\Omega = (-1, 1)^2$, equipped with Dirichlet boundary condition
given by the exact solution.

The exact solution:
\begin{equation}\label{exact-nist-6}
u(x,y) = (1 - e^{-(1 - x) / \epsilon})(1 - e^{-(1 - y) / \epsilon})cos(\pi (x + y)),
\end{equation}
where $\epsilon$ determines the strength of the boundary layer.
The right-hand side $f$ is calculated by inserting (\ref{exact-nist-6}) into (\ref{boundary-layer}).
The solution of NIST-6 with $\epsilon = 10^{-1}$ is shown in Fig. \ref{fig:sln-nist06}.

\begin{figure}[!ht]
\centering
\includegraphics[height=5cm]{nist/nist-6/solution.png}
\caption{The solution to NIST-6 benchmark problem.}
\label{fig:sln-nist06}
\end{figure}

The goal of the benchmark is to reach a relative error below
$10^{-3}$~\% in the $H^1$-norm with as few DOFs as possible.
We begin with adaptive $hp$-FEM with possible anisotropic refinements.
The initial mesh is shown in Fig. \ref{fig:nist-6-hp-aniso} (left).
In a few adaptivity steps, the polynomial degree of elements is increased
anisotropically.
The resulting mesh with 591 DOF is shown in Fig. \ref{fig:nist-6-hp-aniso} (right).

\begin{figure}[!ht]
\centering
\includegraphics[height=5cm]{nist/nist-6/mesh_hp_aniso_init.png}\ \
\includegraphics[height=5cm]{nist/nist-6/mesh_hp_aniso.png}
\vspace{-2mm}
\caption{Initial mesh (left) and final mesh (right) for $hp$-FEM with anisotropic refinements.}
\label{fig:nist-6-hp-aniso}
\end{figure}

The final relative error estimate in $H^1$-norm was 6.23458e-04 \%,
and it was identical to the exact error in all printed digits.
We also solved this benchmark with adaptive $h$-FEM
with linear (left) and quadratic (right)
elements, with anisotropic refinements enabled.
Final meshes for the $h$-FEM computations are shown
in Fig. \ref{fig:nist-6-h-aniso}.

\begin{figure}[!ht]
\centering
\includegraphics[height=5cm]{nist/nist-6/mesh_h1_aniso.png}\ \
\includegraphics[height=5cm]{nist/nist-6/mesh_h2_aniso.png}
\vspace{-2mm}
\caption{Final mesh for $h$-FEM anisotropic refinements with linear and quadratic elements.}
\label{fig:nist-6-h-aniso}
\end{figure}

Figs. \ref{fig:nist-6-conv} compare all
three approaches to automatic adaptivity from the point
of view of DOF and CPU convergence.

\begin{figure}[!ht]
\centering
\includegraphics[height=5cm]{nist/nist-6/conv_dof_aniso.png}\ \
\includegraphics[height=5cm]{nist/nist-6/conv_cpu_aniso.png}
\caption{DOF and CPU time convergence graphs.}
\label{fig:nist-6-conv}
\end{figure}

