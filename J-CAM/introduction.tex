\section{Introduction}
\label{sec:intro}

The number of adaptive Finite Element codes is growing.
Let us mention (in alphabetical order): Alberta
\cite{alberta}, DealII \cite{dealii}, FEniCS
\cite{fenics}, FETK \cite{fetk}, Hermes \cite{hermes},
libMesh \cite{libmesh}, Phaml \cite{phaml}, PHG \cite{phg},
2dhp90 \cite{2dhp90}, and others.
There is no common approach to test adaptive
Finite Element algorithms. The obstacle in developing such an approach is that 
the codes differ in application platforms, ways of loading the physical model, 
grid formats, boundary conditions handling, input/output formats and the list could go on.

Moreover, some Finite Element codes are specifically designed to be used for 
a narrow profile of problems, which also limits the possibility of comparing 
their performance on a single problem.
For instance, MSC.NASTRAN is a Finite Element analysis software,
which is widely used in the aerospace industry, ANSYS, ABAQUS and ADINA are
mostly used in civil engineering, and Ansoft is mainly used in electromagnetic
field analysis, problems of electrical engineering and electronic engineering.
Therefore, in order to compare efficiency and robustness of the Finite Element 
codes, various test problems are required.

For some classical Finite Element problems, pre-processing 
needed for various codes differs greatly. 
As a result of the differences between the codes, 
the same algorithm may give different results and convergency 
behavior for the same problem.

The criterion of performance for adaptive algorithms 
is the obtained accuracy as a function of the total number
of DOFs (Degrees Of Freedom) and CPU time. However, 
this is quite difficult to establish for CPU time because 
the various codes may run on different hardware.

At this point we would like to acknowledge the work of
Dr. William Mitchell (NIST) who collected a suite of
twelve benchmarks for adaptive FEM \cite{mitchell-1}. 
The aim of these is to compare different algorithms 
using a test set of problems with known exact solutions 
that are able to test various aspects of automatic 
adaptivity in the appropriate manner.

In this paper, we solve the twelve benchmarks. 
The test problems and their exact solutions are 
formulated in Sections \ref{sec:bench-1} - \ref{sec:bench-12}. 
We also present solutions obtained by {\sc Hermes} library (http://hpfem.org/hermes). 
{\sc Hermes} is a multi-platform open source C++ 
library for rapid development of adaptive $hp$-FEM
and $hp$-DG solvers. Each section also contains a short 
description of the benchmark problem, then the numerical 
results are presented and discussed. Conclusion and outlooks 
are offered in the last section of the paper.


