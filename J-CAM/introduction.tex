\section{Introduction}
\label{sec:intro}

The number of adaptive Finite Element codes is growing.
Let us mention (in alphabetical order): Alberta
\cite{alberta}, DealII \cite{dealii}, FEniCS
\cite{fenics}, FETK \cite{fetk}, Hermes \cite{hermes}, 
libMesh \cite{libmesh}, Phaml \cite{phaml}, PHG \cite{phg}, 
2dhp90 \cite{2dhp90}, and others.
%A natural question that arises is how they perform compared to each other?
%Unfortunately, comparison efforts are usually inhibited at the very beginning
%by diverse installation requirements, number of supporting libraries, non-unified
%input and output data formats, and different usage of various codes. And even if
%these problems can be overcome, there is not many benchmarks with known exact
%solutions that are able to test various aspects of automatic adaptivity in
%the appropriate manner.
There is no common approach to test all adaptive 
Finite Element algorithms. The obstacle in developing such an approach is that the codes differ in application platforms, ways of loading physical model, grid formats, boundary conditions handling, input/output formats and the list could go on. 

In addition, some Finite Element codes are specifically designed to be used for a narrow profile of problems, which also limits the possibility of comparing their performance.
For instance, MSC.NASTRAN is a universal Finite Element analysis software, 
which is widely used in the aerospace industry, ANSYS, ABAQUS and ADINA are 
mostly used in civil engineering, and Ansoft is mainly used in electromagnetic 
field analysis problems of electrical engineering and electronic engineering. 
Therefore, in order to compare efficiency and robustness of the Finite Element codes, various test problems are required. 

For some classical Finite Element problems, pre-processing needed for various codes differ greatly. As a result of the differences between the codes, the same algorithm may give different results and convergency development for the same problem.

The criterion of performance for adaptive algorithms is the obtained accuracy as a function of the total number
of DOF and CPU time. However, this is quite difficult to establish
for CPU time because the various codes were run on different hardware.
It is common to compare different algorithms using a large test set
with known exact solutions that are able to test various aspects of
automatic adaptivity in the appropriate manner. 
At this point we would like to acknowledge the work of
Dr. William Mitchell (NIST) who collected a suite of
twelve benchmarks for adaptive FEM \cite{mitchell-1}. 

In this paper, we present a series of benchmark problems 
with diverse solutions contained in \cite{mitchell-1}
that are designed to test the ability to handle different 
types of singularities, and other problems an adaptive algorithm must be able to cope with for different equations.

The test problems and their solutions are formulated in Sections
\ref{sec:bench-1} - \ref{sec:bench-12}. We also present solutions obtained by {\sc Hermes} library (http://hpfem.org/hermes). {\sc Hermes} is a multi-platform 
open source C++ library for rapid development of adaptive $hp$-FEM 
and $hp$-DG solvers. Each section begins with a short description
of benchmark problem compared, then the numerical results are
presented and discussed. Conclusion and outlooks are offered
in the last section of the paper.


