\section{Introduction}
\label{sec:intro}

The number of adaptive Finite Element codes is growing.
Let us mention (in alphabetical order): Alberta
\cite{alberta}, DealII \cite{dealii}, FEniCS
\cite{fenics}, FETK \cite{fetk}, Hermes \cite{hermes}, libMesh \cite{libmesh},
Phaml \cite{phaml}, PHG \cite{phg}, 2dhp90 \cite{2dhp90} and many others.\\
%A natural question that arises is how they perform compared to each other?
%Unfortunately, comparison efforts are usually inhibited at the very beginning
%by diverse installation requirements, number of supporting libraries, non-unified
%input and output data formats, and different usage of various codes. And even if
%these problems can be overcome, there is not many benchmarks with known exact
%solutions that are able to test various aspects of automatic adaptivity in
%the appropriate manner.

[Some more descriptions is coming soon.]\\

It is common to compare different softwares using a large test set 
with known exact solutions that are able to test various aspects of 
automatic adaptivity in the appropriate manner. 
At this point we would like to acknowledge the work of 
Dr. William Mitchell (NIST) who collected a suite of 
twelve benchmarks for adaptive FEM \cite{mitchell-1}. \\

In this paper, we presents twelve parametrized test problems 
with diverse solutions contained in \cite{mitchell-1} 
that are designed to test the ability of adaptive algorithms. 
The test problems and their solutions are formulated in Sections 
\ref{sec:bench-1} - \ref{sec:bench-12} by using {\sc Hermes} 
library (http://hpfem.org/hermes). {\sc Hermes} is an open source 
C++ library for rapid development of adaptive $hp$-FEM and 
$hp$-DG solvers. In each section begins with a short description 
of benchmark problem compared, then the numerical results are 
presented and discussed. Conclusion and outlooks are offered 
in the last section of this paper. 
