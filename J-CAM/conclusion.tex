\section{Conclusion and Outlook}
\label{sec:conclusion}

A challenging set of benchmarks aimed at testing adaptive Finite Element Method implementations in terms of handling diverse problems, and diverse obstacles in their solution, has been presented in this paper. 

The solutions and convergence rates obtained by the use of Hermes exemplify that modern adaptive FEM codes can handle a wide range of problems with relative ease. 

Hermes also allowed for the comparison of anisotropic hp-FEM to low order (anisotropic) h-FEM. Results of this comparison furnish evidence that hp-FEM consistently outperforms simple h- adaptivity on all kinds of problems, and that hp-FEM can achieve truly exponential convergence.

The numerical results are given in such a way to make it possible to compare them to results obtained with another implementation of adaptive Finite Element Method. 

In this paper we only solved linear PDE problems where the approximate solution $u$ was a continuous function from the $H^1$ space.
Hermes can also solve equations whose solutions lie in spaces
$Hcurl$, $Hdiv$ or $L^2$, and one can combine these spaces for systems of PFEs.

The computations were performed on a standard laptop, with average performance.

\section{Acknowledgment}

This work was supported by Subcontract No. 00089911 of Battelle
Energy Alliance (DOE intermediary) as well as by the
Grant No. IAA100760702 of the Grant Agency of the Academy
of Sciences of the Czech Republic. The first autor was partly
supported by the National Natural Science Foundation
of China under Projects No. 41074099.
