\section{Benchmark NIST-9 "Wave Front"}
\label{sec:bench-9}

This is a commonly used example for testing the performance of
adaptive refinement algorithms on the wave front and the singularity \cite{mitchell-1, mitchell-2}.
The solution has a sharp circular wave front in the interior of the
domain, with a singularity at the center of the circle.
The equation solved is the Poisson's equation.

\begin{equation} \label{wave-front}
-\Delta u = f
\end{equation}

in the domain $\Omega = (0, 1)^2$, equipped with Dirichlet boundary conditions
given by the exact solution. The exact solution:

\begin{equation}\label{exact-nist-9}
u(x, y) = tan^{-1}(\alpha (r - r_{0}))
\end{equation}

where $r = \sqrt{(x - x_{loc})^{2} + (y - y_{loc})^{2}}$.
Here $(x_{loc}, y_{loc})$ is the center of the circular wave front,
$r_{0}$ is the distance from the wave front to the center of the circle,
and $\alpha$ gives the steepness of the wave front.
The right-hand side $f$ is calculated by inserting (\ref{exact-nist-9}) into (\ref{wave-front}).
The solution of NIST-9 with $\alpha = 50$, $(x_{loc}, y_{loc}) = (0.5, 0.5)$,
$r_{0} = 0.25$ is shown in Fig. \ref{fig:sln-nist09}.

\begin{figure}[!ht]
\centering
\includegraphics[height=6cm]{nist/nist-9/solution.png}
\caption{The solution to NIST-9 benchmark problem.}
\label{fig:sln-nist09}
\end{figure}

The goal of the benchmark is to reach a relative error below
$10^{-1}$~\% in the $H^1$-norm with as few DOFs as possible.
We begin with adaptive $hp$-FEM,
the initial mesh is shown in Fig. \ref{fig:nist-9-hp-aniso} (left).
After 13 adaptivity steps, the resulting mesh with 1465 DOF is shown
in Fig. \ref{fig:nist-9-hp-aniso} (right).

\begin{figure}[!ht]
\centering
\includegraphics[height=5cm]{nist/nist-9/mesh_hp_aniso_init.png}\ \
\includegraphics[height=5cm]{nist/nist-9/mesh_hp_aniso.png}
\caption{Initial mesh (left) and final mesh (right) for $hp$-FEM with anisotropic refinements.}
\label{fig:nist-9-hp-aniso}
\end{figure}

The final relative error estimate in $H^1$-norm was 8.67667e-01 \%,
and it was identical to the exact error in all printed digits.
We also solved this benchmark with adaptive $h$-FEM
with linear (left) and quadratic (right)
elements, with anisotropic refinements enabled.
Final meshes for the $h$-FEM computations are shown
in Fig. \ref{fig:nist-9-h-aniso}.

\begin{figure}[!ht]
\centering
\includegraphics[height=5cm]{nist/nist-9/mesh_h1_aniso.png}\ \
\includegraphics[height=5cm]{nist/nist-9/mesh_h2_aniso.png}
\caption{Final mesh for $h$-FEM anisotropic refinements with linear and quadratic elements.}
\label{fig:nist-9-h-aniso}
\end{figure}

Finally, Figs. \ref{fig:nist-9-conv} compare all
three approaches to automatic adaptivity from the point
of view of DOF and CPU convergence.

\begin{figure}[!ht]
\centering
\includegraphics[height=5cm]{nist/nist-9/conv_dof_aniso.png}\ \
\includegraphics[height=5cm]{nist/nist-9/conv_cpu_aniso.png}
\caption{DOF and CPU time convergence graphs.}
\label{fig:nist-9-conv}
\end{figure}

