%%
%% Copyright 2007, 2008, 2009 Elsevier Ltd
%%
%% This file is part of the 'Elsarticle Bundle'.
%% ---------------------------------------------
%%
%% It may be distributed under the conditions of the LaTeX Project Public
%% License, either version 1.2 of this license or (at your option) any
%% later version.  The latest version of this license is in
%%    http://www.latex-project.org/lppl.txt
%% and version 1.2 or later is part of all distributions of LaTeX
%% version 1999/12/01 or later.
%%
%% The list of all files belonging to the 'Elsarticle Bundle' is
%% given in the file `manifest.txt'.
%%

%% Template article for Elsevier's document class `elsarticle'
%% with numbered style bibliographic references
%% SP 2008/03/01
%%
%%
%%
%% $Id: elsarticle-template-num.tex 4 2009-10-24 08:22:58Z rishi $
%%
%%
\documentclass[12pt]{elsarticle}

%% Use the option review to obtain double line spacing
%% \documentclass[preprint,review,12pt]{elsarticle}

%% Use the options 1p,twocolumn; 3p; 3p,twocolumn; 5p; or 5p,twocolumn
%% for a journal layout:
%% \documentclass[final,1p,times]{elsarticle}
%% \documentclass[final,1p,times,twocolumn]{elsarticle}
%% \documentclass[final,3p,times]{elsarticle}
%% \documentclass[final,3p,times,twocolumn]{elsarticle}
%% \documentclass[final,5p,times]{elsarticle}
%% \documentclass[final,5p,times,twocolumn]{elsarticle}

%% if you use PostScript figures in your article
%% use the graphics package for simple commands
%% \usepackage{graphics}
%% or use the graphicx package for more complicated commands
\usepackage{graphicx}
%% or use the epsfig package if you prefer to use the old commands
%% \usepackage{epsfig}

%% The amssymb package provides various useful mathematical symbols
%\usepackage{amssymb}
%% The amsthm package provides extended theorem environments
%% \usepackage{amsthm}

%% The lineno packages adds line numbers. Start line numbering with
%% \begin{linenumbers}, end it with \end{linenumbers}. Or switch it on
%% for the whole article with \linenumbers after \end{frontmatter}.
%% \usepackage{lineno}

%% natbib.sty is loaded by default. However, natbib options can be
%% provided with \biboptions{...} command. Following options are
%% valid:

%%   round  -  round parentheses are used (default)
%%   square -  square brackets are used   [option]
%%   curly  -  curly braces are used      {option}
%%   angle  -  angle brackets are used    <option>
%%   semicolon  -  multiple citations separated by semi-colon
%%   colon  - same as semicolon, an earlier confusion
%%   comma  -  separated by comma
%%   numbers-  selects numerical citations
%%   super  -  numerical citations as superscripts
%%   sort   -  sorts multiple citations according to order in ref. list
%%   sort&compress   -  like sort, but also compresses numerical citations
%%   compress - compresses without sorting
%%
%% \biboptions{comma,round}

% \biboptions{}

\journal{Computational and Applied Mathematics}

\begin{document}

\begin{frontmatter}

%% Title, authors and addresses

%% use the tnoteref command within \title for footnotes;
%% use the tnotetext command for the associated footnote;
%% use the fnref command within \author or \address for footnotes;
%% use the fntext command for the associated footnote;
%% use the corref command within \author for corresponding author footnotes;
%% use the cortext command for the associated footnote;
%% use the ead command for the email address,
%% and the form \ead[url] for the home page:
%%

\title{A Set of difficulty Problems for Testing\\ Adaptive Finite Element Methods}

%% use optional labels to link authors explicitly to addresses:
\author[label1]{Zhonghua Ma}
\ead{mazhonghua83@gmail.com}
\author[label2]{Lukas Korous}
\ead{lukas.korous@gmail.com}
\author[label3]{Erick Santiago}
\ead{laviticus@sbcglobal.net}
\address[label1]{China University of Petroleum, Beijing, China}
\address[label2]{Charles University, Prague, Czech Republic}
\address[label3]{University of Nevada, Reno, USA}

\begin{abstract}
Adaptive grid refinement is a critical component of algorithms for the numerical solution of partial differential equations (PDEs).
The development of new algorithms and computer
codes for the solution of PDEs usually involves the use of proof-of-concept test problems.

It is common to compare different algorithms using a large test set, to evaluate the algorithm's overall quality, which lies in the ability to handle all kinds of problems, and also to determine the algorithm's strengths and weaknesses.

In this paper we solve a set of benchmark problems devised by
W. Mitchel at NIST \cite{mitchell-1}.
The problems exhibit various types of singularities, disruptions, and oscillations.

Each of the benchmark problem is introduced, together with its exact solution.
Then the solution obtained with the use of the multi-platform open source
C++ library for rapid development of adaptive $hp$-FEM and $hp$-DG solvers {\sc Hermes}\footnote{http://hpfem.org/hermes} is shown,
complemented with convergence graphs and comparison of the fully anisotropic
$hp$-FEM to low-order FEM in terms of convergence.
Overview of Hermes is given in the appendix.
\end{abstract}

\begin{keyword}
$hp$-FEM \sep FEM benchmark \sep anisotropic solution \sep test problems \sep finite element method \sep Hermes \sep Hermes2D
%% keywords here, in the form: keyword \sep keyword
%% MSC codes here, in the form: \MSC code \sep code
%% or \MSC[2008] code \sep code (2000 is the default)
\end{keyword}

\end{frontmatter}

%%
%% Start line numbering here if you want
%%
% \linenumbers

%% main text
\section{Introduction}
\label{sec:intro}

The number of adaptive Finite Element codes is growing.
Let us mention (in alphabetical order): Alberta
\cite{alberta}, DealII \cite{dealii}, FEniCS
\cite{fenics}, FETK \cite{fetk}, Hermes \cite{hermes},
libMesh \cite{libmesh}, Phaml \cite{phaml}, PHG \cite{phg},
2dhp90 \cite{2dhp90}, and others.
There is no common approach to test adaptive
Finite Element algorithms. The obstacle in developing such an approach is that
the codes differ in application platforms, ways of loading the physical model,
grid formats, boundary conditions handling, input/output formats and the list could go on.

Moreover, some Finite Element codes are specifically designed to be used for
a narrow profile of problems, which also limits the possibility of comparing
their performance on a single problem.
For instance, MSC.NASTRAN is a Finite Element analysis software,
which is widely used in the aerospace industry, ANSYS, ABAQUS and ADINA are
mostly used in civil engineering, and Ansoft is mainly used in electromagnetic
field analysis, problems of electrical engineering and electronic engineering.
Therefore, in order to compare efficiency and robustness of the Finite Element
codes, various test problems are required.

For some classical Finite Element problems, pre-processing
needed for various codes differs greatly.
As a result of the differences between the codes,
the same algorithm may give different results and convergency
behavior for the same problem.

The criterion of performance for adaptive algorithms
is the obtained accuracy as a function of the total number
of DOFs (Degrees Of Freedom) and CPU time. However,
this is quite difficult to establish for CPU time because
the various codes may run on different hardware.

At this point we would like to acknowledge the work of
Dr. William Mitchell (NIST) who collected a suite of
twelve benchmarks for adaptive FEM \cite{mitchell-1}.
The aim of these is to compare different algorithms
using a test set of problems with known exact solutions
that are able to test various aspects of automatic
adaptivity in the appropriate manner.

In this paper, we solve the twelve benchmarks.
The test problems and their exact solutions are
formulated in Sections \ref{sec:bench-1} - \ref{sec:bench-12}.
We also present solutions obtained by {\sc Hermes} library (http://hpfem.org/hermes).
{\sc Hermes} is a multi-platform open source C++
library for rapid development of adaptive $hp$-FEM
and $hp$-DG solvers. Each section also contains a short
description of the benchmark problem, then the numerical
results are presented and discussed. Conclusion and outlooks
are offered in the last section of the paper.

%%%%%%%%%%%%%%%%%%%%%%%%%%%%%%%%%%%%%%%%%%%%

\section{Benchmark NIST-1 "Analytic Solution"}
\label{sec:bench-1}

This is the first benchmark problem with a smooth solution
that is used for testing adaptive algorithms performance
where adaptivity isn't really needed.
The equation solved is the Poisson's equation.

\begin{equation} \label{poisson}
-\Delta u = f
\end{equation}
in the domain $\Omega = (0, 1)^2$, equipped with Dirichlet
boundary condition given by the exact solution.
The exact solution $u(x, y) = 2^{4p}x^{p}(1-x)^{p}y^{p}(1-y)^{p}$
of this problem is shown in Fig. \ref{fig:sln-nist01}.
Here $p$ is a parameter, determining the polynomial degree of the exact solution.

\begin{figure}[!ht]
\centering
\includegraphics[height=5cm]{nist/nist-1/solution.png}
\caption{The solution to NIST-1 benchmark problem.}
\label{fig:sln-nist01}
\end{figure}

The goal of the benchmark is to reach a relative error below
$10^{-2}$~\% in the $H^1$-norm with as few degrees of freedom (DOF) as possible.
We begin with the adaptive $hp$-FEM with possible anisotropic refinements (adaptivity mode
HP\_ANISO\_H in {\sc Hermes}). The initial mesh is shown in Fig. \ref{fig:nist-1-hp-aniso} (left).
After 11 adaptivity steps, the resulting mesh with 769 DOF is shown in Fig. \ref{fig:nist-1-hp-aniso} (right).

\begin{figure}[!ht]
\centering
\includegraphics[height=5cm]{nist/nist-1/mesh_hp_aniso_init.png}\ \
\includegraphics[height=5cm]{nist/nist-1/mesh_hp_aniso.png}
%\vspace{-2mm}
\caption{Initial mesh (left) and final mesh (right) for $hp$-FEM with anisotropic refinements.}
\label{fig:nist-1-hp-aniso}
\end{figure}

The final relative error estimate in $H^1$-norm was 4.69543e-03 \%,
and it was identical to the exact error in all printed digits.
We also solved this benchmark with adaptive $h$-FEM
with linear (left) and quadratic (right)
elements, with anisotropic refinements enabled.
Final meshes for the $h$-FEM computations are shown
in Fig. \ref{fig:nist-1-h-aniso}.

\begin{figure}[!ht]
\centering
\includegraphics[height=5cm]{nist/nist-1/mesh_h1_aniso.png}\ \
\includegraphics[height=5cm]{nist/nist-1/mesh_h2_aniso.png}
%\vspace{-2mm}
\caption{Final mesh for $h$-FEM anisotropic refinements with linear and quadratic elements.}
\label{fig:nist-1-h-aniso}
\end{figure}

Finally, Figs. \ref{fig:nist-1-conv} compare all
three approaches to automatic adaptivity from the point
of view of DOF and CPU convergence.

\begin{figure}[!ht]
\centering
\includegraphics[height=5cm]{nist/nist-1/conv_dof_aniso.png}\ \
\includegraphics[height=5cm]{nist/nist-1/conv_cpu_aniso.png}
%\vspace{-2mm}
\caption{DOF and CPU time convergence graphs.}
\label{fig:nist-1-conv}
\end{figure}

%%%%%%%%%%%%%%%%%%%%%%%%%%%%%%%%%%%%%%%%%%%%%%%%%%

\section{Benchmark NIST-2 "Reentrant Corner"}
\label{sec:bench-2}

This is a standard benchmark for adaptive FEM algorithms.
A reentrant corner is nothing more than an internal or inside corner.
It is a very frequent problem and it can cause the error of the solution
calculated by an adaptive method to converge very slowly or not to converge at all.
The exact solution of this problem is smooth but it contains
singular gradient in the reentrant corner.
The equation solved is the Laplace's equation.

\begin{equation} \label{laplace}
-\Delta u = 0
\end{equation}

in the domain $\Omega = (-1, 1)^2$, with a unit square
section removed from the bottom part of the positive $x$ axis.
Equation (\ref{laplace}) equipped with Dirichlet
boundary conditions given by the exact solution:

\begin{equation}\label{exact-nist-2}
u(x, y) = r^{\alpha}\sin(\alpha \theta)
\end{equation}

where $\alpha = \pi / \omega$, $r = \sqrt{x^2+y^2}$,
and $\theta = tan^{-1}(y/x)$. Here $\omega $ determines
the angle of the reentrant corner.
The solution of NIST-2 with $\omega = 3 \pi / 2$
is shown in Fig. \ref{fig:sln-nist02}.

\begin{figure}[!ht]
\centering
\includegraphics[height=5cm]{nist/nist-2/solution.png}
\caption{The solution to NIST-2 benchmark problem.}
\label{fig:sln-nist02}
\end{figure}

The goal of the benchmark is to reach a relative error below
$10^{-1}$~\% in the $H^1$-norm with as few DOFs as possible.
We begin with adaptive $hp$-FEM,
the initial mesh is shown in Fig. \ref{fig:nist-2-hp-aniso} (left).
After 13 adaptivity steps, the resulting mesh with 622 DOF is shown
in Fig. \ref{fig:nist-2-hp-aniso} (right).

\begin{figure}[!ht]
\centering
\includegraphics[height=5cm]{nist/nist-2/mesh_hp_aniso_init.png}\ \
\includegraphics[height=5cm]{nist/nist-2/mesh_hp_aniso.png}
\caption{Initial mesh (left) and final mesh (right) for $hp$-FEM with anisotropic refinements.}
\label{fig:nist-2-hp-aniso}
\end{figure}

The final relative error estimate in $H^1$-norm was 8.15289e-02 \%,
and it was identical to the exact error in all printed digits.
We also solved this benchmark with adaptive $h$-FEM
with linear (left) and quadratic (right)
elements, with anisotropic refinements enabled.
Final meshes for the $h$-FEM computations are shown
in Fig. \ref{fig:nist-2-h-aniso}.

\begin{figure}[!ht]
\centering
\includegraphics[height=5cm]{nist/nist-2/mesh_h1_aniso.png}\ \
\includegraphics[height=5cm]{nist/nist-2/mesh_h2_aniso.png}
\caption{Final mesh for $h$-FEM anisotropic refinements with linear and quadratic elements.}
\label{fig:nist-2-h-aniso}
\end{figure}

Finally, Figs. \ref{fig:nist-2-conv} compare all
three approaches to automatic adaptivity from the point
of view of DOF and CPU convergence.

\begin{figure}[!ht]
\centering
\includegraphics[height=5cm]{nist/nist-2/conv_dof_aniso.png}\ \
\includegraphics[height=5cm]{nist/nist-2/conv_cpu_aniso.png}
\caption{DOF and CPU time convergence graphs.}
\label{fig:nist-2-conv}
\end{figure}

%%%%%%%%%%%%%%%%%%%%%%%%%%%%%%%%%%%%%%%%%%%%%%%%%%%%%%%

\section{Benchmark NIST-3 "Linear Elasticity"}
\label{sec:bench-3}

Linear elasticity is used extensively in structural analysis
and engineering. Linear elasticity is a simplification
of the general nonlinear equations of elasticity and is the mathematical
study of how solid objects deform and become internally
stressed due to prescribed loading conditions.
In this benchmark, we present a standard system of two
coupled equations with mixed derivative for linear elasticity
in the coupling term. This example employs the adaptive multimesh $hp$-FEM
to solve the equations.

\begin{equation}\label{crack}
\left\{
\begin{array}{l}
\displaystyle
-E \frac{1-\nu^2}{1-2\nu} \frac{\partial^{2} u}{\partial x^{2}} - E\frac{1-\nu^2}{2-2 \nu} \frac{\partial^{2} u}{\partial y^{2}}
-E \frac{1-\nu^2}{(1-2\nu)(2-2\nu)} \frac{\partial^{2} v}{\partial x \partial y} = F_{x} \\
\displaystyle
-E \frac{1-\nu^2}{2-2\nu} \frac{\partial^{2} v}{\partial x^{2}} - E\frac{1-\nu^2}{1-2\nu} \frac{\partial^{2} v}{\partial y^{2}}
-E \frac{1-\nu^2}{(1-2\nu)(2-2\nu)} \frac{\partial^{2} u}{\partial x \partial y} = F_{y}
\end{array}
\right.
\end{equation}

where $F_{x} = F_{y} = 0$, $u$ and $v$ are the
$x$ and $y$ displacements, $E$ is Young's Modulus,
and $\nu$ is Poisson's ratio.

The domain in the example is $\Omega = (-1, 1)^2$ with a slit,
equipped with Dirichlet boundary conditions given by the
exact solution. The exact solution of (\ref{crack}) in polar coordinates is given by

\begin{equation}\label{exact-nist-3}
\left\{
\begin{array}{l}
\displaystyle
u(x, y) = \frac{1}{2G} r^{\lambda}[(k - Q(\lambda + 1))cos(\lambda \theta) - \lambda cos((\lambda - 2) \theta)] \\
\displaystyle
v(x, y) = \frac{1}{2G} r^{\lambda}[(k + Q(\lambda + 1))sin(\lambda \theta) + \lambda sin((\lambda - 2) \theta)]
\end{array}
\right.
\end{equation}

here $\lambda = 0.5444837367825$, $Q = 0.5430755788367$,
$k = 3 - 4 \nu$ and $G = E / (2(1 + \nu))$.
The solution of NIST-3 is shown in Fig. \ref{fig:sln-nist03}.

\begin{figure}[!ht]
\centering
\includegraphics[height=40mm]{nist/nist-3/solution-u.png}\ \
\includegraphics[height=40mm]{nist/nist-3/solution-v.png}
%\vspace{-2mm}
\caption{The $u$ (left) and $v$ (right) component to NIST-3 benchmark problem.}
\label{fig:sln-nist03}
\end{figure}

The goal of the benchmark is to reach a relative error below
$10^{-1}$~\% in the $H^1$-norm with as few DOFs as possible.
We begin with adaptive $hp$-FEM,
the initial mesh is shown in Fig. \ref{fig:nist-3-hp-aniso-init}.

\begin{figure}[!ht]
\centering
\includegraphics[height=5cm]{nist/nist-3/mesh_hp_aniso_init.png}
\caption{Initial mesh.}
\label{fig:nist-3-hp-aniso-init}
\end{figure}

After 40 adaptivity steps,
the resulting mesh with 3897 DOF is shown in Fig. \ref{fig:nist-3-hp-aniso}.

\begin{figure}[!ht]
\centering
\includegraphics[height=5cm]{nist/nist-3/mesh_u_hp_anisoh.png}\ \
\includegraphics[height=5cm]{nist/nist-3/mesh_v_hp_anisoh.png}
\caption{Final meshs of $u$ (left) and $v$ (right) component for $hp$-FEM with anisotropic refinements.}
\label{fig:nist-3-hp-aniso}
\end{figure}

The final relative error estimate in $H^1$-norm was 8.05635e-02 \%,
and it was identical to the exact error in all printed digits.
We also solved this benchmark with adaptive $h$-FEM
with linear and quadratic elements, with anisotropic refinements enabled.
Final meshes for the $h$-FEM computations are shown
in Fig. \ref{fig:nist-3-h1-aniso} and Fig. \ref{fig:nist-3-h2-aniso}.

\begin{figure}[!ht]
\centering
\includegraphics[height=5cm]{nist/nist-3/mesh_u_h1_aniso.png}\ \
\includegraphics[height=5cm]{nist/nist-3/mesh_v_h1_aniso.png}
\caption{Final meshs of $u$ (left) and $v$ (right) component for $h$-FEM anisotropic refinements with linear elements.}
\label{fig:nist-3-h1-aniso}
\end{figure}

\begin{figure}[!ht]
\centering
\includegraphics[height=5cm]{nist/nist-3/mesh_u_h2_aniso.png}\ \
\includegraphics[height=5cm]{nist/nist-3/mesh_v_h2_aniso.png}
\caption{Final meshs of $u$ (left) and $v$ (right) component for $h$-FEM anisotropic refinements with quadratic elements.}
\label{fig:nist-3-h2-aniso}
\end{figure}

Finally, Figs. \ref{fig:nist-3-conv} compare all
three approaches to automatic adaptivity from the point
of view of DOF and CPU convergence.

\begin{figure}[!ht]
\centering
\includegraphics[height=5cm]{nist/nist-3/conv_dof_aniso.png}\ \
\includegraphics[height=5cm]{nist/nist-3/conv_cpu_aniso.png}
\caption{DOF and CPU time convergence graphs.}
\label{fig:nist-3-conv}
\end{figure}

%%%%%%%%%%%%%%%%%%%%%%%%%%%%%%%%%%%%%%%%%%%%%%%%%%%%%%%%%%%%

\section{Benchmark NIST-4 "Peak"}
\label{sec:bench-4}

This problem has an exponential peak in the interior of the domain.
The equation solved in this benchmark problem is the Poisson's equation.

\begin{equation} \label{poisson-peak}
-\Delta u = f
\end{equation}

in the domain $\Omega = (0, 1)^2$, equipped with Dirichlet
boundary conditions given by the exact solution.
The exact solution is

\begin{equation}\label{exact-nist-4}
u(x,y) = e^{-\alpha ((x - x_{loc})^{2} + (y - y_{loc})^{2})}
\end{equation}

where $(x_{loc}, y_{loc})$ is the location of the peak,
and $\alpha$ determines the strength of the peak.
The right-hand side $f$ is calculated by inserting (\ref{exact-nist-4}) into (\ref{poisson-peak}).
The solution of NIST-4 with $\alpha = 1000$,
$(x_{loc}, y_{loc}) = (0.5, 0.5)$ is shown in Fig. \ref{fig:sln-nist04}.

\begin{figure}[!ht]
\centering
\includegraphics[height=5cm]{nist/nist-4/solution.png}
\caption{The solution to NIST-4 benchmark problem.}
\label{fig:sln-nist04}
\end{figure}

The goal of the benchmark is to reach a relative error below
$10^{-2}$~\% in the $H^1$-norm with as few DOFs as possible.
We begin with adaptive $hp$-FEM,
the initial mesh is shown in Fig. \ref{fig:nist-4-hp-aniso} (left).
After 12 adaptivity steps, the resulting mesh with 1561 DOF is shown
in Fig. \ref{fig:nist-4-hp-aniso} (right).

\begin{figure}[!ht]
\centering
\includegraphics[height=5cm]{nist/nist-4/mesh_hp_aniso_init.png}\ \
\includegraphics[height=5cm]{nist/nist-4/mesh_hp_aniso.png}
\caption{Initial mesh (left) and final mesh (right) for $hp$-FEM with anisotropic refinements.}
\label{fig:nist-4-hp-aniso}
\end{figure}

The final relative error estimate in $H^1$-norm was 7.02865e-03 \%,
and it was identical to the exact error in all printed digits.
We also solved this benchmark with adaptive $h$-FEM
with linear (left) and quadratic (right)
elements, with anisotropic refinements enabled.
Final meshes for the $h$-FEM computations are shown
in Fig. \ref{fig:nist-4-h-aniso}.

\begin{figure}[!ht]
\centering
\includegraphics[height=5cm]{nist/nist-4/mesh_h1_aniso.png}\ \
\includegraphics[height=5cm]{nist/nist-4/mesh_h2_aniso.png}
\caption{Final mesh for $h$-FEM anisotropic refinements with linear and quadratic elements.}
\label{fig:nist-4-h-aniso}
\end{figure}

Finally, Figs. \ref{fig:nist-4-conv} compare all
three approaches to automatic adaptivity from the point
of view of DOF and CPU convergence.

\begin{figure}[!ht]
\centering
\includegraphics[height=5cm]{nist/nist-4/conv_dof_aniso.png}\ \
\includegraphics[height=5cm]{nist/nist-4/conv_cpu_aniso.png}
\caption{DOF and CPU time convergence graphs.}
\label{fig:nist-4-conv}
\end{figure}

%%%%%%%%%%%%%%%%%%%%%%%%%%%%%%%%%%%%%%%%%%%%%%%%%%%%%%%%%%%

\section{Benchmark NIST-5 "Battery"}
\label{sec:bench-5}

This is a heat conduction problem in a nonhomogeneous material.
It comes with an anisotropic solution with strong internal boundary
layers and multiple singularities.
The solution has multiple point singularities in the interior at which
more than three different materials meet which are stronger than those
corresponding to reentrant corners \cite{demkowicz-1}.
The equation solved is given by

\begin{equation} \label{heat-conduction}
-\frac{\partial }{\partial x}\left(p(x, y)\frac{\partial u}{\partial x}\right)
-\frac{\partial }{\partial y}\left(q(x, y)\frac{\partial u}{\partial y}\right) = f
\end{equation}

in the domain $\Omega = (0, 8.4) \times (0, 24)$, equipped with
zero Neumann boundary condition on the left edge, Natural boundary
conditions on the rest of the boundary:

\begin{equation}
\left\{
\begin{array}{l}
\displaystyle p(x, y)\frac{\partial u}{\partial x}\nu_1 + q(x, y)\frac{\partial u}{\partial y}\nu_2 = g_{left}(x, y) \ \mbox{on} \  \Gamma_{left} \\
\displaystyle p(x, y)\frac{\partial u}{\partial x}\nu_1 + q(x, y)\frac{\partial u}{\partial y}\nu_2 + c(x, y)u = g_{right}(x, y) \ \mbox{on} \ \Gamma_{right} \\
\displaystyle p(x, y)\frac{\partial u}{\partial x}\nu_1 + q(x, y)\frac{\partial u}{\partial y}\nu_2 + c(x, y)u = g_{top}(x, y) \ \mbox{on} \ \Gamma_{top} \\
\displaystyle p(x, y)\frac{\partial u}{\partial x}\nu_1 + q(x, y)\frac{\partial u}{\partial y}\nu_2 + c(x, y)u = g_{bottom}(x, y) \ \mbox{on} \ \Gamma_{bottom}
\end{array}
\right.
\end{equation}

where $p(x, y)$, $q(x, y)$, $c(x, y)$, $g(x, y)$, and the right hand
side $f$ are constant functions (different in respective materials).
The solution of NIST-5 is shown in Fig. \ref{fig:sln-nist05}.

\begin{figure}[!ht]
\centering
\includegraphics[height=5cm]{nist/nist-5/solution.png}
\caption{The solution to NIST-5 benchmark problem.}
\label{fig:sln-nist05}
\end{figure}

The goal of the benchmark is to reach a relative error below
$1.5e-02$~\% in the $H^1$-norm with as few DOFs as possible.
We begin with adaptive $hp$-FEM,
the initial mesh is shown in Fig. \ref{fig:nist-5-hp-aniso} (left).
After 32 adaptivity steps, the resulting mesh with 7450 DOF is shown
in Fig. \ref{fig:nist-5-hp-aniso} (right).

\begin{figure}[!ht]
\centering
\includegraphics[height=5cm]{nist/nist-5/mesh_hp_aniso_init.png}\ \
\includegraphics[height=5cm]{nist/nist-5/mesh_hp_aniso.png}
\vspace{-2mm}
\caption{Initial mesh (left) and final mesh (right) for $hp$-FEM with anisotropic refinements.}
\label{fig:nist-5-hp-aniso}
\end{figure}

The final relative error estimate in $H^1$-norm was 1.46775e-02 \%.
We also solved this benchmark with adaptive $h$-FEM
with linear (left) and quadratic (right)
elements, with anisotropic refinements enabled.
Final meshes for the $h$-FEM computations are shown
in Fig. \ref{fig:nist-5-h-aniso}.

\begin{figure}[!ht]
\centering
\includegraphics[height=5cm]{nist/nist-5/mesh_h1_aniso.png}\ \
\includegraphics[height=5cm]{nist/nist-5/mesh_h2_aniso.png}
\caption{Final mesh for $h$-FEM anisotropic refinements with linear and quadratic elements.}
\label{fig:nist-5-h-aniso}
\end{figure}

Finally, Figs. \ref{fig:nist-5-conv} compare all
three approaches to automatic adaptivity from the point
of view of DOF and CPU convergence.

\begin{figure}[!ht]
\centering
\includegraphics[height=5cm]{nist/nist-5/conv_dof_aniso.png}\ \
\includegraphics[height=5cm]{nist/nist-5/conv_cpu_aniso.png}
\caption{DOF and CPU time convergence graphs.}
\label{fig:nist-5-conv}
\end{figure}

%%%%%%%%%%%%%%%%%%%%%%%%%%%%%%%%%%%%%%%%%%%%%%%%

\section{Benchmark NIST-6 "Boundary Layer"}
\label{sec:bench-6}

This example is a singularly perturbed problem with known exact solution that exhibits
a boundary layer along the right and top sides of the domain.
It is a convection-diffusion equation with first order terms.

\begin{equation} \label{boundary-layer}
-\epsilon \nabla^{2} u + 2\frac{\partial u}{\partial x} + \frac{\partial u}{\partial y}= f
\end{equation}

in the domain $\Omega = (-1, 1)^2$, equipped with Dirichlet boundary condition
given by the exact solution. The exact solution:

\begin{equation}\label{exact-nist-6}
u(x,y) = (1 - e^{-(1 - x) / \epsilon})(1 - e^{-(1 - y) / \epsilon})cos(\pi (x + y))
\end{equation}

where $\epsilon$ determines the strength of the boundary layer.
The right-hand side $f$ is calculated by inserting (\ref{exact-nist-6}) into (\ref{boundary-layer}).
The solution of NIST-6 containing a boundary layer
with $\epsilon = 10^{-1}$ is shown in Fig. \ref{fig:sln-nist06}.

\begin{figure}[!ht]
\centering
\includegraphics[height=5cm]{nist/nist-6/solution.png}
\caption{The solution to NIST-6 benchmark problem.}
\label{fig:sln-nist06}
\end{figure}

The goal of the benchmark is to reach a relative error below
$10^{-3}$~\% in the $H^1$-norm with as few DOFs as possible.
We begin with adaptive $hp$-FEM,
the initial mesh is shown in Fig. \ref{fig:nist-6-hp-aniso} (left).
After 18 adaptivity steps, the resulting mesh with 591 DOF is shown
in Fig. \ref{fig:nist-6-hp-aniso} (right).

\begin{figure}[!ht]
\centering
\includegraphics[height=5cm]{nist/nist-6/mesh_hp_aniso_init.png}\ \
\includegraphics[height=5cm]{nist/nist-6/mesh_hp_aniso.png}
\caption{Initial mesh (left) and final mesh (right) for $hp$-FEM with anisotropic refinements.}
\label{fig:nist-6-hp-aniso}
\end{figure}

The final relative error estimate in $H^1$-norm was 6.23458e-04 \%,
and it was identical to the exact error in all printed digits.
We also solved this benchmark with adaptive $h$-FEM
with linear (left) and quadratic (right)
elements, with anisotropic refinements enabled.
Final meshes for the $h$-FEM computations are shown
in Fig. \ref{fig:nist-6-h-aniso}.

\begin{figure}[!ht]
\centering
\includegraphics[height=5cm]{nist/nist-6/mesh_h1_aniso.png}\ \
\includegraphics[height=5cm]{nist/nist-6/mesh_h2_aniso.png}
\caption{Final mesh for $h$-FEM anisotropic refinements with linear and quadratic elements.}
\label{fig:nist-6-h-aniso}
\end{figure}

Finally, Figs. \ref{fig:nist-6-conv} compare all
three approaches to automatic adaptivity from the point
of view of DOF and CPU convergence.

\begin{figure}[!ht]
\centering
\includegraphics[height=5cm]{nist/nist-6/conv_dof_aniso.png}\ \
\includegraphics[height=5cm]{nist/nist-6/conv_cpu_aniso.png}
\caption{DOF and CPU time convergence graphs.}
\label{fig:nist-6-conv}
\end{figure}

%%%%%%%%%%%%%%%%%%%%%%%%%%%%%%%%%%%%%%%%%%%%%%%%%%%%%%

\section{Benchmark NIST-7 "Boundary Line Singularity"}
\label{sec:bench-7}

This is a singularity problem with the solution that is singular along the left part of the boundary.
The equation solved in this problem is the Poisson's equation.

\begin{equation} \label{boundary-line-singularity}
-\Delta u = f
\end{equation}

in the domain $\Omega = (0, 1)^2$, equipped with Dirichlet boundary conditions
given by the exact solution. The exact solution:

\begin{equation}\label{exact-nist-7}
u(x,y) = x^{\alpha}
\end{equation}

where $\alpha \geq 0.5$ determines the strength of the singularity.
The right-hand side $f$ is calculated by inserting (\ref{exact-nist-7}) into (\ref{boundary-line-singularity}).
The solution of NIST-7 with $\alpha = 0.6$ is shown in Fig. \ref{fig:sln-nist07}.

\begin{figure}[!ht]
\centering
\includegraphics[height=6cm]{nist/nist-7/solution.png}
\caption{The solution to NIST-7 benchmark problem.}
\label{fig:sln-nist07}
\end{figure}

The goal of the benchmark is to reach a relative error below
$1.5$~\% in the $H^1$-norm with as few DOFs as possible.
We begin with adaptive $hp$-FEM,
the initial mesh is shown in Fig. \ref{fig:nist-7-hp-aniso} (left).
After 43 adaptivity steps, the resulting mesh with 88 DOF is shown
in Fig. \ref{fig:nist-7-hp-aniso} (right).

\begin{figure}[!ht]
\centering
\includegraphics[height=5cm]{nist/nist-7/mesh_hp_aniso_init.png}\ \
\includegraphics[height=5cm]{nist/nist-7/mesh_hp_aniso.png}
\caption{Initial mesh (left) and final mesh (right) for $hp$-FEM with anisotropic refinements.}
\label{fig:nist-7-hp-aniso}
\end{figure}

The final relative error estimate in $H^1$-norm was 1.46348 \%,
and it was identical to the exact error in all printed digits.
We also solved this benchmark with adaptive $h$-FEM
with linear (left) and quadratic (right)
elements, with anisotropic refinements enabled.
Final meshes for the $h$-FEM computations are shown
in Fig. \ref{fig:nist-7-h-aniso}.

\begin{figure}[!ht]
\centering
\includegraphics[height=5cm]{nist/nist-7/mesh_h1_aniso.png}\ \
\includegraphics[height=5cm]{nist/nist-7/mesh_h2_aniso.png}
\caption{Final mesh for $h$-FEM anisotropic refinements with linear and quadratic elements.}
\label{fig:nist-7-h-aniso}
\end{figure}

Finally, Figs. \ref{fig:nist-7-conv} compare all
three approaches to automatic adaptivity from the point
of view of DOF and CPU convergence.

\begin{figure}[!ht]
\centering
\includegraphics[height=5cm]{nist/nist-7/conv_dof_aniso.png}\ \
\includegraphics[height=5cm]{nist/nist-7/conv_cpu_aniso.png}
\caption{DOF and CPU time convergence graphs.}
\label{fig:nist-7-conv}
\end{figure}

%%%%%%%%%%%%%%%%%%%%%%%%%%%%%%%%%%%%%%%%%%%%%%%%%%%

\section{Benchmark NIST-8 "Oscillatory"}
\label{sec:bench-8}

This is a wave function that satisfies the Schr\"{o}dinger's equation model of two
interacting atoms, highly oscillatory near the origin.
The equation solved in this problem is Helmholtz equation.

\begin{equation} \label{oscillatory}
-\nabla^{2} u - \frac{1}{(\alpha + r)^{4}} u = f
\end{equation}

in the domain $\Omega = (0, 1)^2$, equipped with Dirichlet boundary conditions
given by the exact solution. The exact solution:

\begin{equation}\label{exact-nist-8}
u(x,y) = sin(\frac{1}{\alpha + r})
\end{equation}

where $r = \sqrt{x^{2} + y^{2}}$, $\alpha = \frac{1}{N \pi}$ determines the number of oscillations.
The right-hand side $f$ is calculated by inserting (\ref{exact-nist-8}) into (\ref{oscillatory}).
The solution of NIST-8 with $\alpha = \frac{1}{10 \pi}$ is shown in Fig. \ref{fig:sln-nist08}.

\begin{figure}[!ht]
\centering
\includegraphics[height=6cm]{nist/nist-8/solution.png}
\caption{The solution to NIST-8 benchmark problem.}
\label{fig:sln-nist08}
\end{figure}

The goal of the benchmark is to reach a relative error below
$10^{-2}$~\% in the $H^1$-norm with as few DOFs as possible.
We begin with adaptive $hp$-FEM,
the initial mesh is shown in Fig. \ref{fig:nist-8-hp-aniso} (left).
After 28 adaptivity steps, the resulting mesh with 1160 DOF is shown
in Fig. \ref{fig:nist-8-hp-aniso} (right).

\begin{figure}[!ht]
\centering
\includegraphics[height=5cm]{nist/nist-8/mesh_hp_aniso_init.png}\ \
\includegraphics[height=5cm]{nist/nist-8/mesh_hp_aniso.png}
\vspace{-2mm}
\caption{Initial mesh (left) and final mesh (right) for $hp$-FEM with anisotropic refinements.}
\label{fig:nist-8-hp-aniso}
\end{figure}

The final relative error estimate in $H^1$-norm was 9.09848e-02 \%,
and it was identical to the exact error in all printed digits.
We also solved this benchmark with adaptive $h$-FEM
with linear (left) and quadratic (right)
elements, with anisotropic refinements enabled.
Final meshes for the $h$-FEM computations are shown
in Fig. \ref{fig:nist-8-h-aniso}.

\begin{figure}[!ht]
\centering
\includegraphics[height=5cm]{nist/nist-8/mesh_h1_aniso.png}\ \
\includegraphics[height=5cm]{nist/nist-8/mesh_h2_aniso.png}
\vspace{-2mm}
\caption{Final mesh for $h$-FEM anisotropic refinements with linear and quadratic elements.}
\label{fig:nist-8-h-aniso}
\end{figure}

Finally, Figs. \ref{fig:nist-8-conv} compare all
three approaches to automatic adaptivity from the point
of view of DOF and CPU convergence.

\begin{figure}[!ht]
\centering
\includegraphics[height=5cm]{nist/nist-8/conv_dof_aniso.png}\ \
\includegraphics[height=5cm]{nist/nist-8/conv_cpu_aniso.png}
\caption{DOF and CPU time convergence graphs.}
\label{fig:nist-8-conv}
\end{figure}

%%%%%%%%%%%%%%%%%%%%%%%%%%%%%%%%%%%%%%%%%%%%%%%%%%%

\section{Benchmark NIST-9 "Wave Front"}
\label{sec:bench-9}

This is a commonly used example for testing the performance of
adaptive refinement algorithms on the wave front and the singularity \cite{mitchell-1, mitchell-2}.
The solution has a sharp circular wave front in the interior of the
domain, with a singularity at the center of the circle.
The equation solved is the Poisson's equation.

\begin{equation} \label{wave-front}
-\Delta u = f
\end{equation}

in the domain $\Omega = (0, 1)^2$, equipped with Dirichlet boundary conditions
given by the exact solution. The exact solution:

\begin{equation}\label{exact-nist-9}
u(x, y) = tan^{-1}(\alpha (r - r_{0}))
\end{equation}

where $r = \sqrt{(x - x_{loc})^{2} + (y - y_{loc})^{2}}$.
Here $(x_{loc}, y_{loc})$ is the center of the circular wave front,
$r_{0}$ is the distance from the wave front to the center of the circle,
and $\alpha$ gives the steepness of the wave front.
The right-hand side $f$ is calculated by inserting (\ref{exact-nist-9}) into (\ref{wave-front}).
The solution of NIST-9 with $\alpha = 50$, $(x_{loc}, y_{loc}) = (0.5, 0.5)$,
$r_{0} = 0.25$ is shown in Fig. \ref{fig:sln-nist09}.

\begin{figure}[!ht]
\centering
\includegraphics[height=6cm]{nist/nist-9/solution.png}
\caption{The solution to NIST-9 benchmark problem.}
\label{fig:sln-nist09}
\end{figure}

The goal of the benchmark is to reach a relative error below
$10^{-1}$~\% in the $H^1$-norm with as few DOFs as possible.
We begin with adaptive $hp$-FEM,
the initial mesh is shown in Fig. \ref{fig:nist-9-hp-aniso} (left).
After 13 adaptivity steps, the resulting mesh with 1465 DOF is shown
in Fig. \ref{fig:nist-9-hp-aniso} (right).

\begin{figure}[!ht]
\centering
\includegraphics[height=5cm]{nist/nist-9/mesh_hp_aniso_init.png}\ \
\includegraphics[height=5cm]{nist/nist-9/mesh_hp_aniso.png}
\caption{Initial mesh (left) and final mesh (right) for $hp$-FEM with anisotropic refinements.}
\label{fig:nist-9-hp-aniso}
\end{figure}

The final relative error estimate in $H^1$-norm was 8.67667e-01 \%,
and it was identical to the exact error in all printed digits.
We also solved this benchmark with adaptive $h$-FEM
with linear (left) and quadratic (right)
elements, with anisotropic refinements enabled.
Final meshes for the $h$-FEM computations are shown
in Fig. \ref{fig:nist-9-h-aniso}.

\begin{figure}[!ht]
\centering
\includegraphics[height=5cm]{nist/nist-9/mesh_h1_aniso.png}\ \
\includegraphics[height=5cm]{nist/nist-9/mesh_h2_aniso.png}
\caption{Final mesh for $h$-FEM anisotropic refinements with linear and quadratic elements.}
\label{fig:nist-9-h-aniso}
\end{figure}

Finally, Figs. \ref{fig:nist-9-conv} compare all
three approaches to automatic adaptivity from the point
of view of DOF and CPU convergence.

\begin{figure}[!ht]
\centering
\includegraphics[height=5cm]{nist/nist-9/conv_dof_aniso.png}\ \
\includegraphics[height=5cm]{nist/nist-9/conv_cpu_aniso.png}
\caption{DOF and CPU time convergence graphs.}
\label{fig:nist-9-conv}
\end{figure}

%%%%%%%%%%%%%%%%%%%%%%%%%%%%%%%%%%%%%%%%%%%%%%

\section{Benchmark NIST-10 "Interior Line Singularity"}
\label{sec:bench-10}

This is another example with anisotropic solution that is suitable for testing
anisotropic element refinements. The equation solved is the Poisson's equation.

\begin{equation} \label{interior}
-\Delta u = f
\end{equation}

in the domain $\Omega = (-1, 1)^2$, equipped with a zero
Neumann boundary condition on left edge, Dirichlet boundary
conditions given by the exact solution on the rest of the boundary.
The exact solution:

\begin{equation}\label{exact-nist-10}
\left\{
\begin{array}{l}
\displaystyle
u(x,y) = \cos(Ky)\ \ \ \mbox{for}\ x \le 0 \\
u(x,y) = \cos(Ky) + x^{\alpha}\ \ \ \mbox{for}\ x > 0
\end{array}
\right.
\end{equation}

where $K$ and $\alpha$ are real constants.
The right-hand side $f$ is calculated by inserting
(\ref{exact-nist-10}) into (\ref{interior}).
The solution of NIST-10 containing a line singularity with $K = \pi/2$ and
$\alpha = 2.01$ is shown in Fig. \ref{fig:sln-nist10}.

\begin{figure}[!ht]
\centering
\includegraphics[height=5cm]{nist/nist-10/solution.png}
\caption{The solution to NIST-10 benchmark problem.}
\label{fig:sln-nist10}
\end{figure}

The goal of the benchmark is to reach a relative error below
$10^{-4}$~\% in the $H^1$-norm with as few DOFs as possible.
We begin with adaptive $hp$-FEM,
the initial mesh is shown in Fig. \ref{fig:nist-10-hp-aniso} (left).
After 14 adaptivity steps, the resulting mesh with 381 DOF is shown
in Fig. \ref{fig:nist-10-hp-aniso} (right).

\begin{figure}[!ht]
\centering
\includegraphics[height=5cm]{nist/nist-10/mesh_hp_aniso_init.png}\ \
\includegraphics[height=5cm]{nist/nist-10/mesh_hp_aniso.png}
\vspace{-2mm}
\caption{Initial mesh (left) and final mesh (right) for $hp$-FEM with anisotropic refinements.}
\label{fig:nist-10-hp-aniso}
\end{figure}

The final relative error estimate in $H^1$-norm was 8.68994e-05 \%,
and it was identical to the exact error in all printed digits.
We also solved this benchmark with adaptive $h$-FEM
with linear (left) and quadratic (right)
elements, with anisotropic refinements enabled.
Final meshes for the $h$-FEM computations are shown
in Fig. \ref{fig:nist-10-h-aniso}.

\begin{figure}[!ht]
\centering
\includegraphics[height=5cm]{nist/nist-10/mesh_h1_aniso.png}\ \
\includegraphics[height=5cm]{nist/nist-10/mesh_h2_aniso.png}
\vspace{-2mm}
\caption{Final mesh for $h$-FEM anisotropic refinements with linear and quadratic elements.}
\label{fig:nist-10-h-aniso}
\end{figure}

Finally, Figs. \ref{fig:nist-10-conv} compare all
three approaches to automatic adaptivity from the point
of view of DOF and CPU convergence.

\begin{figure}[!ht]
\centering
\includegraphics[height=5cm]{nist/nist-10/conv_dof_aniso.png}\ \
\includegraphics[height=5cm]{nist/nist-10/conv_cpu_aniso.png}
\caption{DOF and CPU time convergence graphs.}
\label{fig:nist-10-conv}
\end{figure}

%%%%%%%%%%%%%%%%%%%%%%%%%%%%%%%%%%%%%%%%%%%%%%%%%%%%

\section{Benchmark NIST-11 "Intersecting Interfaces"}
\label{sec:bench-11}

This is a Poisson problem with intersecting interfaces,
divide the plane into four regions.
The solution to this problem contains a severe
singularity that poses a challenge to adaptive methods.
The equation solved is given by

\begin{equation} \label{intersecting}
-\nabla \cdot (a(x,y) \nabla u) = 0
\end{equation}

where the parameter $a$ is piecewise-constant,
$a(x,y) = 161.4476387975881$ in the first and third quadrants,
and $a(x,y) = 1$ in the remaining two quadrants.
The domain of this problem is $\Omega = (-1, 1)^2$, equipped with
Dirichlet boundary conditions given by the exact solution.
The exact solution:

\begin{equation}\label{exact-nist-11}
u(x,y) = r^{a_1} \mu (\theta)
\end{equation}

where $a_1$ and $\mu (\theta)$ is given in \cite{mitchell-1}.
The right-hand side $f$ is calculated by inserting
(\ref{exact-nist-11}) into (\ref{intersecting}).
The solution of NIST-11 is shown in Fig. \ref{fig:sln-nist11}.

\begin{figure}[!ht]
\centering
\includegraphics[height=5cm]{nist/nist-11/solution.png}
\caption{The solution to NIST-11 benchmark problem.}
\label{fig:sln-nist11}
\end{figure}

The goal of the benchmark is to reach a relative error below
$0.5$~\% in the $H^1$-norm with as few DOFs as possible.
We begin with adaptive $hp$-FEM,
the initial mesh is shown in Fig. \ref{fig:nist-11-hp-aniso} (left).
After 46 adaptivity steps, the resulting mesh with 591 DOF is shown
in Fig. \ref{fig:nist-11-hp-aniso} (right).

\begin{figure}[!ht]
\centering
\includegraphics[height=5cm]{nist/nist-11/mesh_hp_aniso_init.png}\ \
\includegraphics[height=5cm]{nist/nist-11/mesh_hp_aniso.png}
\caption{Initial mesh (left) and final mesh (right) for $hp$-FEM with anisotropic refinements.}
\label{fig:nist-11-hp-aniso}
\end{figure}

The final relative error estimate in $H^1$-norm was 4.7087e-01 \%,
and it was identical to the exact error in all printed digits.
We also solved this benchmark with adaptive $h$-FEM
with linear (left) and quadratic (right)
elements, with anisotropic refinements enabled.
Final meshes for the $h$-FEM computations are shown
in Fig. \ref{fig:nist-11-h-aniso}.

\begin{figure}[!ht]
\centering
\includegraphics[height=5cm]{nist/nist-11/mesh_h1_aniso.png}\ \
\includegraphics[height=5cm]{nist/nist-11/mesh_h2_aniso.png}
\caption{Final mesh for $h$-FEM anisotropic refinements with linear and quadratic elements.}
\label{fig:nist-11-h-aniso}
\end{figure}

Finally, Figs. \ref{fig:nist-11-conv} compare all
three approaches to automatic adaptivity from the point
of view of DOF and CPU convergence.

\begin{figure}[!ht]
\centering
\includegraphics[height=5cm]{nist/nist-11/conv_dof_aniso.png}\ \
\includegraphics[height=5cm]{nist/nist-11/conv_cpu_aniso.png}
\caption{DOF and CPU time convergence graphs.}
\label{fig:nist-11-conv}
\end{figure}

%%%%%%%%%%%%%%%%%%%%%%%%%%%%%%%%%%%%%%%%%%%%%%%%%

\section{Benchmark NIST-12 "Multiple Difficulties"}
\label{sec:bench-12}

The solution to this problem combines four types of benchmarks
seen in previous sections into the same problem,
where the wave front intersects the boundary
layer and corner singularity, and the peak is centered on the wave front.
%It contains a point singularity due to a reentrant corner, a circular wave front (which might
%include a singularity at the center of the circle), a sharp peak, and a boundary layer.
The equation solved is the Poisson's equation.

\begin{equation} \label{multiple}
-\Delta u = f
\end{equation}

in the L-shaped domain, equipped with Dirichlet boundary conditions
given by the exact solution.
The exact solution:

\begin{equation}\label{exact-nist-12}
u(x,y) =  r^{\alpha_{C} }\sin(\alpha_{C} \theta)
+ e^{-\alpha_{P} ((x - x_{P})^{2} + (y - y_{P})^{2})}
+ tan^{-1}(\alpha_{W} (r_{W} - r_{0}))
+ e^{-(1 - y) / \epsilon}
\end{equation}

where $\alpha_C = \pi / \omega_C$, $r = \sqrt{x^2+y^2}$
and $\theta = tan^{-1}(y/x)$, here $\omega_C$ determines
the angle of the re-entrant corner.
$(x_{P}, y_{P})$ is the location of the peak, $\alpha$
determines the strength of the peak. Furthermore
$r_{W} = \sqrt{(x - x_{W})^{2} + (y - y_{W})^{2}}$,
where $(x_{W}, y_{W})$ is the center of the circular wave front,
$r_{0}$ is the distance from the wave front to the
center of the circle, and $\alpha_W$ gives
the steepness of the wave front. The parameter $\epsilon$ determines the
strength of the boundary layer, the boundary layer was placed on $y = -1$.
The right-hand side $f$ is calculated by inserting (\ref{exact-nist-12})
into (\ref{multiple}).
The solution of NIST-12 with $\omega_C = 3 \pi /2$,
$(x_{W}, y_{W}) = (0, -3/4)$, $r_{0} = 3/4$, $\alpha_{W} = 200$,
$(x_{P}, y_{P}) = (\sqrt{5} / 4, -1/4)$,
$\epsilon = 1/100$ is shown in Fig. \ref{fig:sln-nist12}.

\begin{figure}[!ht]
\centering
\includegraphics[height=5cm]{nist/nist-12/solution.png}
\caption{The solution to NIST-12 benchmark problem.}
\label{fig:sln-nist12}
\end{figure}

The goal of the benchmark is to reach a relative error below
$10^{-1}$~\% in the $H^1$-norm with as few DOFs as possible.
We begin with adaptive $hp$-FEM,
the initial mesh is shown in Fig. \ref{fig:nist-12-hp-aniso} (left).
After 26 adaptivity steps, the resulting mesh with 4438 DOF is shown
in Fig. \ref{fig:nist-12-hp-aniso} (right).

\begin{figure}[!ht]
\centering
\includegraphics[height=5cm]{nist/nist-12/mesh_hp_aniso_init.png}\ \
\includegraphics[height=5cm]{nist/nist-12/mesh_hp_aniso.png}
\caption{Initial mesh (left) and final mesh (right) for $hp$-FEM with anisotropic refinements.}
\label{fig:nist-12-hp-aniso}
\end{figure}

The final relative error estimate in $H^1$-norm was 9.85118e-01 \%,
and it was identical to the exact error in all printed digits.
We also solved this benchmark with adaptive $h$-FEM
with linear (left) and quadratic (right)
elements, with anisotropic refinements enabled.
Final meshes for the $h$-FEM computations are shown
in Fig. \ref{fig:nist-12-h-aniso}.

\begin{figure}[!ht]
\centering
\includegraphics[height=5cm]{nist/nist-12/mesh_h1_aniso.png}\ \
\includegraphics[height=5cm]{nist/nist-12/mesh_h2_aniso.png}
\caption{Final mesh for $h$-FEM anisotropic refinements with linear and quadratic elements.}
\label{fig:nist-12-h-aniso}
\end{figure}

Finally, Figs. \ref{fig:nist-12-conv} compare all
three approaches to automatic adaptivity from the point
of view of DOF and CPU convergence.

\begin{figure}[!ht]
\centering
\includegraphics[height=5cm]{nist/nist-12/conv_dof_aniso.png}\ \
\includegraphics[height=5cm]{nist/nist-12/conv_cpu_aniso.png}
\caption{DOF and CPU time convergence graphs.}
\label{fig:nist-12-conv}
\end{figure}


\section{Conclusion and Outlook}
\label{sec:conclusion}

A set of benchmarks that can be used to assess the
ability of adaptive FEM codes to handle for diverse
problems have been presented in this paper.
Numerical results with twelve test problems demonstrate
that ......

So far we have just solved linear PDE problems where $u$
was a continuous approximation in the $H^1$ space.
Hermes can also solve equations whose solutions lie in the space
$Hcurl$, $Hdiv$ or $L^2$ and one can combine these spaces for PDE systems.

The computations were performed on an Intel Core i3 based laptop PC
operating under the 32 bit VMware Fedora 12 distribution
of Linux with kernel 2.6.31.5-127.fc12.i686.PAE
on Windows XP professional(1GB memory for VMware Workstation).

The obtained rate confirms the theoretical estimates.
Notice that the hp strategy consistently outperforms the h adaptivity.

The purpose of this papaer is to ......(coming soon.)

\section{Acknowledgment}

This work was supported by Subcontract No. 00089911 of Battelle
Energy Alliance (DOE intermediary) as well as by the
Grant No. IAA100760702 of the Grant Agency of the Academy
of Sciences of the Czech Republic. The first autor was partly
supported by the National Natural Science Foundation
of China under Projects No. 41074099.


\begin{thebibliography}{[KLR73]}

%\bibitem{label1}
%D. Estep, G. Carey, V. Ginting, S. Tavener, T. Wildey:
%A Posteriori Error Analysis of Multiscale Operator
%Decomposition Methods for Multiphysics Models, SciDAC 2008,
%Journal of Physics: Conference Series 125 (2008) 12 - 75.
%
%\bibitem{fitzhugh}
%R. Fitzhugh: Mathematical Models of Excitation and Propagation in Nerve.
%Chapter 1, pp. 1-85 in H.P. Schwan, ed. Biological Engineering,
%McGraw-Hill Book Co., N.Y. (1969).

\bibitem{mitchell-1}
W. Mitchell: A Collection of 2D Elliptic Problems for
Testing Adaptive Algorithms, NISTIR 7668, 2010 (available online).

\bibitem{mitchell-2}
W. Mitchell: A Survey of hp-Adaptive Strategies for Elliptic Partial Differential Equations,
Annals of the European Academy of Sciences, to appear (available online).

\bibitem{demkowicz-1}
L. Demkowicz: One and Two Dimensional Elliptic and Maxwell Problems,
Chapman \& Hall \/ CRC Press, Taylor \& Francis, 2006.

%\bibitem{nagumo}
%J. Nagumo, S. Arimoto, S. Yoshizawa:
%An Active Pulse Transmission Line Simulating Nerve Axon. Proc. IRE 50, 2061–2070 (1962).

\bibitem{label2}
P. Solin, D. Andrs, J. Cerveny, M. Simko:
PDE-Independent Adaptive $hp$-FEM Based on Hierarchic Extension of Finite Element Spaces.
J. Comput. Appl. Math. 233 (2010) 3086-3094.

\bibitem{thermoel}
P. Solin, J. Cerveny, L. Dubcova, D. Andrs:
Monolithic Discretization of Linear Thermoelasticity Problems
via Adaptive Multimesh $hp$-FEM, J. Comput. Appl. Math 234 (2010) 2350 - 2357.

\bibitem{sosedo}
P. Solin. K. Segeth, I. Dolezel: Higher-Order Finite Element Methods, Chapman \& Hall
/ CRC Press, Boca Raton, 2003.
\end{thebibliography}

\vbox{}
\vspace{6mm}
We also include online references to the FEM codes mentioned in this paper.
All these URLs were active on January 1, 2011:

\begin{thebibliography}{[codes]}

\bibitem{alberta}
Alberta: http://www.alberta-fem.de/

\bibitem{dealii}
DealII (Differential Equations Analysis Library) \\ http://www.dealii.org/

\bibitem{fenics}
FEniCS: http://www.fenics.org/wiki/FEniCS\_Project

\bibitem{fetk}
FETK (Finite Element Toolkit): http://www.fetk.org/

\bibitem{hermes}
Hermes (Higher-Order Finite Element System) \\ http://hpfem.org/hermes.

\bibitem{libmesh}
libMesh: http://libmesh.sourceforge.net/

\bibitem{phaml}
Phaml (Parallel Hierarchical Adaptive MultiLevel Project): \\ http://math.nist.gov/phaml/

\bibitem{phg}
PHG (Parallel Hierarchical Grid) http://lsec.cc.ac.cn/phg/

\bibitem{2dhp90}
2dhp90: http://users.ices.utexas.edu/\~{}leszek/2dhp90.html

\end{thebibliography}

\section{Appendix 1 - Automatic hp-adaptivity with the open source library Hermes}

\paragraph{Mature hp-adaptivity algorithms}
Hermes puts a major emphasis on error control and automatic adaptivity. Practitioners know well how painful it is to use automatic adaptivity in conjunction with standard lower-order approximations such as linear or quadratic elements - the error decreases somehow during a few initial adaptivity steps, but then it slows down and it does not help to invest more unknowns or CPU time. This is typical for low-order methods. In contrast to this, the exponentially-convergent adaptive hp-FEM and hp-DG do not have this problem - the error drops steadily and fast during adaptivity all the way to the desired accuracy.

\paragraph{Wide applicability}
Hermes is PDE-independent. Standard FEM codes are designed to solve some narrow class(es) of PDE problems (such as elliptic equations, fluid dynamics, electromagnetics etc.). Hermes does not employ any technique or algorithm that would limit its applicability to some particular class(es) of PDE problems. Automatic adaptivity is guided by a universal computational a-posteriori error estimate that works in the same way for any PDE. Visit the hp-FEM group home page and especially the gallery to see numerous examples.

\paragraph{Arbitrary-level hanging nodes}
Hermes has a unique original methodology for handling irregular meshes with arbitrary-level hanging nodes. This means that extremely small elements can be adjacent to very large ones. When an element is refined, its neighbors are never split forcefully as in conventional adaptivity algorithms. It is well known that approximations with one-level hanging nodes are more efficient compared to regular meshes. However, the technique of arbitrary-level hanging nodes brings this to a perfection.

\paragraph{Multimesh hp-FEM}
Various physical fields or solution components in multiphysics problems can be approximated on individual meshes, combining quality $H^1, H(curl), H(div), and L^2$ conforming higher-order elements. Due to a unique original methodology, no error is caused by operator splitting, transferring data between different meshes, and the like. The following figure illustrates a coupled problem of heat and moisture transfer in massive concrete walls of a nuclear reactor vessel.

\begin{figure}[!ht]
\centering
\includegraphics[height=5cm]{img/hermes_hm_sol.png}
\hspace{10mm}
\includegraphics[height=5cm]{img/hermes_hm_mesh.png}
\caption{Illustration of multimesh hp-FEM.}
\label{fig:hermes_hm}
\end{figure}
\noindent

\paragraph{Dynamical meshes for time-dependent problems}
In time-dependent problems, different physical fields or solution components can be approximated on individual meshes that evolve in time independently of each other. Due to a unique original methodology, no error is caused by transfering solution data between different meshes and time levels. No such transfer takes place in the multimesh hp-FEM - the discretization of the time-dependent PDE system is monolithic.


%% Authors are advised to submit their bibtex database files. They are
%% requested to list a bibtex style file in the manuscript if they do
%% not want to use elsarticle-num.bst.

%% References without bibTeX database:

% \begin{thebibliography}{00}

%% \bibitem must have the following form:
%%   \bibitem{key}...
%%

% \bibitem{}

% \end{thebibliography}


\end{document}

%%
%% End of file `elsarticle-template-num.tex'.
