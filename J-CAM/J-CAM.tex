%%
%% Copyright 2007, 2008, 2009 Elsevier Ltd
%%
%% This file is part of the 'Elsarticle Bundle'.
%% ---------------------------------------------
%%
%% It may be distributed under the conditions of the LaTeX Project Public
%% License, either version 1.2 of this license or (at your option) any
%% later version.  The latest version of this license is in
%%    http://www.latex-project.org/lppl.txt
%% and version 1.2 or later is part of all distributions of LaTeX
%% version 1999/12/01 or later.
%%
%% The list of all files belonging to the 'Elsarticle Bundle' is
%% given in the file `manifest.txt'.
%%

%% Template article for Elsevier's document class `elsarticle'
%% with numbered style bibliographic references
%% SP 2008/03/01
%%
%%
%%
%% $Id: elsarticle-template-num.tex 4 2009-10-24 08:22:58Z rishi $
%%
%%
\documentclass[12pt]{elsarticle}

%% Use the option review to obtain double line spacing
%% \documentclass[preprint,review,12pt]{elsarticle}

%% Use the options 1p,twocolumn; 3p; 3p,twocolumn; 5p; or 5p,twocolumn
%% for a journal layout:
%% \documentclass[final,1p,times]{elsarticle}
%% \documentclass[final,1p,times,twocolumn]{elsarticle}
%% \documentclass[final,3p,times]{elsarticle}
%% \documentclass[final,3p,times,twocolumn]{elsarticle}
%% \documentclass[final,5p,times]{elsarticle}
%% \documentclass[final,5p,times,twocolumn]{elsarticle}

%% if you use PostScript figures in your article
%% use the graphics package for simple commands
%% \usepackage{graphics}
%% or use the graphicx package for more complicated commands
\usepackage{graphicx}
\usepackage{float}
%% or use the epsfig package if you prefer to use the old commands
%% \usepackage{epsfig}

%% The amssymb package provides various useful mathematical symbols
%\usepackage{amssymb}
%% The amsthm package provides extended theorem environments
%% \usepackage{amsthm}

%% The lineno packages adds line numbers. Start line numbering with
%% \begin{linenumbers}, end it with \end{linenumbers}. Or switch it on
%% for the whole article with \linenumbers after \end{frontmatter}.
%% \usepackage{lineno}

%% natbib.sty is loaded by default. However, natbib options can be
%% provided with \biboptions{...} command. Following options are
%% valid:

%%   round  -  round parentheses are used (default)
%%   square -  square brackets are used   [option]
%%   curly  -  curly braces are used      {option}
%%   angle  -  angle brackets are used    <option>
%%   semicolon  -  multiple citations separated by semi-colon
%%   colon  - same as semicolon, an earlier confusion
%%   comma  -  separated by comma
%%   numbers-  selects numerical citations
%%   super  -  numerical citations as superscripts
%%   sort   -  sorts multiple citations according to order in ref. list
%%   sort&compress   -  like sort, but also compresses numerical citations
%%   compress - compresses without sorting
%%
%% \biboptions{comma,round}

% \biboptions{}

\journal{Computational and Applied Mathematics}

\begin{document}

\begin{frontmatter}

%% Title, authors and addresses

%% use the tnoteref command within \title for footnotes;
%% use the tnotetext command for the associated footnote;
%% use the fnref command within \author or \address for footnotes;
%% use the fntext command for the associated footnote;
%% use the corref command within \author for corresponding author footnotes;
%% use the cortext command for the associated footnote;
%% use the ead command for the email address,
%% and the form \ead[url] for the home page:
%%

\title{Solving a Suite of NIST Benchmark Problems\\ for Adaptive FEM with the Hermes Library}

%% use optional labels to link authors explicitly to addresses:
\author[label1]{Zhonghua Ma}
\ead{mazhonghua83@gmail.com}
\author[label2]{Lukas Korous}
\ead{lukas.korous@gmail.com}
\author[label3]{Erick Santiago}
\ead{laviticus@sbcglobal.net}
\address[label1]{China University of Petroleum, Beijing, China}
\address[label2]{Charles University, Prague, Czech Republic}
\address[label3]{University of Nevada, Reno, USA}

\begin{abstract}
%Adaptive grid refinement is a critical component of algorithms for the numerical solution of partial 
%differential equations (PDEs). The development of new algorithms and computer
%codes for the solution of PDEs usually involves the use of proof-of-concept test problems.
%It is common to compare different algorithms using a large test set, to evaluate the algorithm's 
%overall quality, which lies in the ability to handle all kinds of problems, and also to determine 
%the algorithm's strengths and weaknesses.
Recently, a new suite of 12 benchmark problems for adaptive finite element methods (FEM)
was published at the U.S. National Institute for Standards and Technology (NIST).
These test problems come with exact solutions, and they exhibit various difficulties 
including singularities, disruptions, and oscillations. They are a great basis for comparison 
of performance of adaptive FEM codes. In this paper we present results obtained
with the open source library Hermes (http://hpfem.org). These results are reproducible --
all of them are part of the Git repository of the Hermes project and the reader can experiment 
with them at will. We hope that authors of other adaptive FEM codes, both academic and 
commercial, will present their results for these test problems in a reproducible fashion as well. 
\end{abstract}

\begin{keyword}
finite element method \sep automatic adaptivity \sep benchmark problem \sep exact solution
%% keywords here, in the form: keyword \sep keyword
%% MSC codes here, in the form: \MSC code \sep code
%% or \MSC[2008] code \sep code (2000 is the default)
\end{keyword}

\end{frontmatter}

%% main text
\section{Introduction}
\label{sec:intro}

The number of adaptive finite element (FEM) codes is growing very fast. 
They differ in deployment operating systems and hardware platforms, 
ways of loading the physical model, error estimation mechanisms, 
algebraic solvers they use, mesh formats, boundary conditions 
handling, input/output formats, and other aspects. Some of them are 
designed for a narrow class of problems while others are supposed to 
cover several physical applications. These facts, and others not listed
here, make it extremely difficult to assess and compare their accuracy and performance.

Recently, Dr. William Mitchell (NIST) collected a suite of
twelve benchmark problems with known exact solution for adaptive 
FEM \cite{mitchell-1}. The advantage of having an exact solution is that 
the approximation error can be calculated very accurately, thus 
enabling fair comparison of results calculated with different
programs. All these examples are elliptic, and they are defined  
in very simple geometries, to make their solution possible with 
virtually any FEM code. 

In this paper, we solve the twelve benchmarks using 
Hermes, a multi-platform open source C++
library for rapid development of adaptive $hp$-FEM
and $hp$-DG solvers (http://hpfem.org). A few words
and links to the Hermes library will be presented at the 
end as this is not the major objective of the paper.
Our goal is to show the results and make them available 
to be reproduced and compared with other results.
 
To make the paper reasonably self-contained, we must repeat some 
material from \cite{mitchell-1}, but this is done in a minimalistic way.
Each of the following sections contains a short description of 
the benchmark problem followed by results and convergence graphs in terms of degrees
of freedom (DOF) and CPU time. Conclusion and outlooks
are presented at the end of the paper.

%%%%%%%%%%%%%%%%%%%%%%%%%%%%%%%%%%%%%%%%%%%%

\section{Benchmark NIST-1 "Analytic Solution"}
\label{sec:bench-1}

This benchmark problem has a smooth solution. 
Solved is the Poisson equation

\begin{equation} \label{poisson}
-\Delta u = f
\end{equation}
in the domain $\Omega = (0, 1)^2$, equipped with Dirichlet
boundary condition given by the exact solution.
The exact solution is $u(x, y) = 2^{4p}x^{p}(1-x)^{p}y^{p}(1-y)^{p}$
with $p = XXX$.
The solution to NIST-1 is shown in Fig. \ref{fig:sln-nist01}.

\begin{figure}[H]
\centering
\vspace{-3mm}
\includegraphics[height=5cm]{nist/nist-1/solution.png}
\vspace{-3mm}
\caption{Solution to the NIST-1 benchmark problem.}
\label{fig:sln-nist01}
\end{figure}

%\begin{figure}[H]
%\centering
%\vspace{-8mm}
%\includegraphics[height=4.7cm]{nist/nist-1/mesh_h1_aniso.png}\ \
%\includegraphics[height=4.7cm]{nist/nist-1/mesh_h2_aniso.png}
%\vspace{-3mm}
%\caption{
%Final mesh (left) with 51365 DOF and the relative error is 5.87223e-01 \% for $h$-FEM with linear elements.
%Final mesh (right) with 43401 DOF and the relative error is 1.37127e-02 \% for $h$-FEM with quadratic elements.}
%\label{fig:nist-1-h-aniso}
%\end{figure}

Fig. \ref{fig:nist-1-hp-aniso} presents the final meshes corresponding to adaptive $h$-FEM with 
linear elements, adaptive $h$-FEM with quadratic elements, and adaptive $hp$-FEM. Different 
polynomial degrees of elements are represented by different colors. 

\begin{figure}[!ht]
\centering
%\vspace{-7mm}
\includegraphics[height=3.7cm]{nist/nist-1/mesh_h1_aniso.png}
\includegraphics[height=3.7cm]{nist/nist-1/mesh_h2_aniso.png}
\includegraphics[height=3.7cm]{nist/nist-1/mesh_hp_aniso.png}
\vspace{-3mm}
\caption{
Left: Final mesh with 51365 DOF and relative error 5.87223e-01~\% for $h$-FEM with linear elements.
Middle: Final mesh with 43401 DOF and relative error 1.37127e-02~\% for $h$-FEM with quadratic elements. 
Right: Final mesh with 769 DOF and relative error 4.69543e-03~\% for $hp$-FEM (adaptivity option HP\_ANISO\_H).}
\label{fig:nist-1-hp-aniso}
\vspace{-3mm}
\end{figure}

Fig. \ref{fig:nist-1-conv} shows the convergence of the adaptive methods in terms of DOF and CPU-time.

\begin{figure}[H]
\centering
\vspace{-3mm}
\hspace{-50mm}
\includegraphics[width=7.5cm]{nist/nist-1/conv_dof_aniso.png}\ \
\hspace{-8mm}
\includegraphics[width=7.5cm]{nist/nist-1/conv_cpu_aniso.png}
\hspace{-50mm}
\vspace{-2mm}
\caption{DOF and CPU time convergence graphs.}
\vspace{-5mm}
\label{fig:nist-1-conv}
\end{figure}

%%%%%%%%%%%%%%%%%%%%%%%%%%%%%%%%%%%%%%%%%%%%%%%%%%

\section{Benchmark NIST-2 "Reentrant Corner"}
\label{sec:bench-2}

The exact solution of this problem is smooth but it contains
singular gradient at the reentrant corner.
Solved is the Laplace equation

\begin{equation} \label{laplace}
-\Delta u = 0
\end{equation}
in the domain $\Omega = (-1, 1)^2$, with a unit square
section removed from the bottom part of the positive $x$ axis.
Equation (\ref{laplace}) equipped with Dirichlet
boundary conditions given by the exact solution
$u(x, y) = r^{\alpha}\sin(\alpha \theta)$,
where $\alpha = \pi / \omega$, $r = \sqrt{x^2+y^2}$,
and $\theta = tan^{-1}(y/x)$. Here $\omega $ determines
the angle of the reentrant corner.
The solution to NIST-2 with $\omega = 3 \pi / 2$
is shown in Fig. \ref{fig:sln-nist02}.

\begin{figure}[H]
\centering
\vspace{-3mm}
\includegraphics[height=5cm]{nist/nist-2/solution.png}
\vspace{-3mm}
\caption{Solution to the NIST-2 benchmark problem.}
\label{fig:sln-nist02}
\end{figure}

%\begin{figure}[H]
%\centering
%\includegraphics[height=5cm]{nist/nist-2/mesh_h1_aniso.png}
%\includegraphics[height=5cm]{nist/nist-2/mesh_h2_aniso.png}
%\caption{
%Final mesh (left) with 46097 DOF and the relative error is 1.30193e-01 \% for $h$-FEM with linear elements.
%Final mesh (right) with 1289 DOF and the relative error is 8.26847e-02 \% for $h$-FEM with quadratic elements.}
%\label{fig:nist-2-h-aniso}
%\end{figure}

Fig. \ref{fig:nist-2-hp-aniso} presents the final meshes corresponding to adaptive $h$-FEM with 
linear elements, adaptive $h$-FEM with quadratic elements, and adaptive $hp$-FEM. Different 
polynomial degrees of elements are represented by different colors. 

\begin{figure}[H]
\centering
\vspace{-5mm}
\includegraphics[height=3.7cm]{nist/nist-2/mesh_h1_aniso.png}
\includegraphics[height=3.7cm]{nist/nist-2/mesh_h2_aniso.png}
\includegraphics[height=3.7cm]{nist/nist-2/mesh_hp_aniso.png}
\vspace{-3mm}
\caption{
Left: Final mesh with 46097 DOF and relative error 1.30193e-01~\% for $h$-FEM with linear elements.
Middle: Final mesh with 1289 DOF and relative error 8.26847e-02~\% for $h$-FEM with quadratic elements. 
Right: Final mesh with 622 DOF and relative error 8.15289e-02~\% for $hp$-FEM (adaptivity option HP\_ANISO\_H).}
\label{fig:nist-2-hp-aniso}
\vspace{-3mm}
\end{figure}

Fig. \ref{fig:nist-2-conv} shows the convergence of the adaptive methods in terms of DOF and CPU-time.

%\vspace{-5mm}

\begin{figure}[H]
\centering
\hspace{-50mm}
\includegraphics[width=7.5cm]{nist/nist-2/conv_dof_aniso.png}\ \
\hspace{-10mm}
\includegraphics[width=7.5cm]{nist/nist-2/conv_cpu_aniso.png}
\hspace{-50mm}
\caption{DOF and CPU time convergence graphs.}
\label{fig:nist-2-conv}
\end{figure}

%%%%%%%%%%%%%%%%%%%%%%%%%%%%%%%%%%%%%%%%%%%%%%%%%%%%%%%

\section{Benchmark NIST-3 "Linear Elasticity"}
\label{sec:bench-3}

%Linear elasticity is used extensively in structural analysis
%and engineering. Linear elasticity is a simplification
%of the general nonlinear equations of elasticity and is the mathematical
%study of how solid objects deform and become internally
%stressed due to prescribed loading conditions.
This benchmark deals with the Lam\'e equations of linear elasticity.
Since the two resulting displacement components $u, v$ are very different, we use the 
multimesh functionality of Hermes \cite{label2,thermoel} which makes it 
possible to approximate them on different meshes that moreover 
are locally refined independently of each other.
The equations have the form

\begin{equation}\label{crack}
\left\{
\begin{array}{l}
\displaystyle
-E \frac{1-\nu^2}{1-2\nu} \frac{\partial^{2} u}{\partial x^{2}} - E\frac{1-\nu^2}{2-2 \nu} \frac{\partial^{2} u}{\partial y^{2}}
-E \frac{1-\nu^2}{(1-2\nu)(2-2\nu)} \frac{\partial^{2} v}{\partial x \partial y} = 0, \\[3mm]
\displaystyle
-E \frac{1-\nu^2}{2-2\nu} \frac{\partial^{2} v}{\partial x^{2}} - E\frac{1-\nu^2}{1-2\nu} \frac{\partial^{2} v}{\partial y^{2}}
-E \frac{1-\nu^2}{(1-2\nu)(2-2\nu)} \frac{\partial^{2} u}{\partial x \partial y} = 0,
\end{array}
\right.
\end{equation}
where $u$ and $v$ are the
$x$ and $y$ displacements, $E$ is Young's Modulus,
and $\nu$ is Poisson ratio.
The domain in the example is $\Omega = (-1, 1)^2$ with a slit $([0,0], [1,0])$,
equipped with Dirichlet boundary conditions given by the
exact solution in polar coordinates as

\[
\left\{
\begin{array}{l}
\displaystyle
u(x, y) = \frac{1}{2G} r^{\lambda}[(k - Q(\lambda + 1))cos(\lambda \theta) - \lambda cos((\lambda - 2) \theta)],  \\[3mm]
\displaystyle
v(x, y) = \frac{1}{2G} r^{\lambda}[(k + Q(\lambda + 1))sin(\lambda \theta) + \lambda sin((\lambda - 2) \theta)],
\end{array}
\right.
\]
where $\lambda = 0.5444837367825$, $Q = 0.5430755788367$,
$k = 3 - 4 \nu$ and $G = E / (2(1 + \nu))$.
The solution to NIST-3 is shown in Fig. \ref{fig:sln-nist03}.
%\vspace{-4mm}

\begin{figure}[H]
\centering
%\vspace{-3mm}
\includegraphics[height=5cm]{nist/nist-3/solution-u.png}\ \
\includegraphics[height=5cm]{nist/nist-3/solution-v.png}
%\vspace{-3mm}
\caption{The $u$ (left) and $v$ (right) component to NIST-3 benchmark problem.}
%\vspace{-3mm}
\label{fig:sln-nist03}
\end{figure}

%\begin{figure}[H]
%\centering
%\vspace{-3mm}
%\includegraphics[height=5cm]{nist/nist-3/mesh_u_h1_aniso.png}
%\includegraphics[height=5cm]{nist/nist-3/mesh_u_h2_aniso.png}
%\includegraphics[height=5cm]{nist/nist-3/mesh_u_hp_anisoh.png}\ \
%\includegraphics[height=5cm]{nist/nist-3/mesh_v_h1_aniso.png}
%\includegraphics[height=5cm]{nist/nist-3/mesh_v_h2_aniso.png}
%\includegraphics[height=5cm]{nist/nist-3/mesh_v_hp_anisoh.png}
%\vspace{-3mm}
%\caption{
%Final mesh (left) with 39779 DOF and the resulting
%relative error estimate in $H^1$-norm of 3.84929e-01 \% for $h$-FEM with linear elements.
%Final mesh (right) with 9330 DOF and the resulting
%relative error estimate in $H^1$-norm of 9.56383e-02 \% for $h$-FEM with quadratic elements.}
%\vspace{-4mm}
%\label{fig:nist-3-h-aniso}
%\end{figure}

Fig. \ref{fig:nist-3-hp-aniso} presents the final meshes corresponding to adaptive $h$-FEM with 
linear elements, adaptive $h$-FEM with quadratic elements, and adaptive $hp$-FEM. Different 
polynomial degrees of elements are represented by different colors. 

\begin{figure}[!ht]
\centering
\includegraphics[height=3.7cm]{nist/nist-3/mesh_u_h1_aniso.png}
\includegraphics[height=3.7cm]{nist/nist-3/mesh_u_h2_aniso.png}
\includegraphics[height=3.7cm]{nist/nist-3/mesh_u_hp_anisoh.png}\ \
\includegraphics[height=3.7cm]{nist/nist-3/mesh_v_h1_aniso.png}
\includegraphics[height=3.7cm]{nist/nist-3/mesh_v_h2_aniso.png}
\includegraphics[height=3.7cm]{nist/nist-3/mesh_v_hp_anisoh.png}
\caption{
Left: Final mesh with 39779 DOF and relative error 3.84929e-01~\% for $h$-FEM with linear elements.
Middle: Final mesh with 9330 DOF and relative error 9.56383e-02~\% for $h$-FEM with quadratic elements. 
Right: Final mesh with 3897 DOF and relative error 8.05635e-02~\% for $hp$-FEM (adaptivity option HP\_ANISO\_H).}
\label{fig:nist-3-hp-aniso}
\end{figure}

Fig. \ref{fig:nist-3-conv} shows the convergence of the adaptive methods in terms of DOF and CPU-time.

\begin{figure}[H]
\centering
\hspace{-50mm}
\includegraphics[width=7.5cm]{nist/nist-3/conv_dof_aniso.png}\ \
\hspace{-10mm}
\includegraphics[width=7.5cm]{nist/nist-3/conv_cpu_aniso.png}
\hspace{-50mm}
\caption{DOF and CPU time convergence graphs.}
\label{fig:nist-3-conv}
\end{figure}

%%%%%%%%%%%%%%%%%%%%%%%%%%%%%%%%%%%%%%%%%%%%%%%%

\section{Benchmark NIST-4 "Peak"}
\label{sec:bench-4}

The solution to this problem exhibits an exponential peak in the interior of the domain.
The equation solved in this benchmark problem is the Poisson equation.

\begin{equation} \label{poisson-peak}
-\Delta u = f
\end{equation}
in the domain $\Omega = (0, 1)^2$, equipped with Dirichlet
boundary conditions given by the exact solution.
The exact solution is
$u(x,y) = e^{-\alpha ((x - x_{loc})^{2} + (y - y_{loc})^{2})}$,
where $(x_{loc}, y_{loc})$ is the location of the peak,
and $\alpha$ determines the strength of the peak.
The solution to NIST-4 with $\alpha = 1000$,
$(x_{loc}, y_{loc}) = (0.5, 0.5)$ is shown in Fig. \ref{fig:sln-nist04}.

\begin{figure}[H]
\centering
%\vspace{-3mm}
\includegraphics[height=5cm]{nist/nist-4/solution.png}
%\vspace{-3mm}
\caption{Solution to the NIST-4 benchmark problem.}
\vspace{-3mm}
\label{fig:sln-nist04}
\end{figure}
%\noindent
%where $(x_{loc}, y_{loc})$ is the location of the peak,
%and $\alpha$ determines the strength of the peak.
%The solution of NIST-4 with $\alpha = 1000$,
%$(x_{loc}, y_{loc}) = (0.5, 0.5)$ is shown in Fig. \ref{fig:sln-nist04}.

%\begin{figure}[H]
%\centering
%\includegraphics[height=5cm]{nist/nist-4/mesh_h1_aniso.png}
%\includegraphics[height=5cm]{nist/nist-4/mesh_h2_aniso.png}
%\caption{
%Final mesh (left) with 58253 DOF and the relative error is 5.72234e-01 \% for $h$-FEM with linear elements.
%Final mesh (right) with 51473 DOF and the relative error is 1.39525e-02 \% for $h$-FEM with quadratic elements.}
%\label{fig:nist-4-h-aniso}
%\end{figure}

Fig. \ref{fig:nist-4-hp-aniso} presents the final meshes corresponding to adaptive $h$-FEM with 
linear elements, adaptive $h$-FEM with quadratic elements, and adaptive $hp$-FEM. Different 
polynomial degrees of elements are represented by different colors. 

\begin{figure}[H]
\centering
\includegraphics[height=3.7cm]{nist/nist-4/mesh_h1_aniso.png}
\includegraphics[height=3.7cm]{nist/nist-4/mesh_h2_aniso.png}
\includegraphics[height=3.7cm]{nist/nist-4/mesh_hp_aniso.png}
\caption{
Left: Final mesh with 58253 DOF and relative error 5.72234e-01~\% for $h$-FEM with linear elements.
Middle: Final mesh with 51473 DOF and relative error 1.39525e-02~\% for $h$-FEM with quadratic elements. 
Right: Final mesh with 1561 DOF and relative error 7.02865e-03~\% for $hp$-FEM (adaptivity option HP\_ANISO\_H).}
\label{fig:nist-4-hp-aniso}
\end{figure}

Fig. \ref{fig:nist-4-conv} shows the convergence of the adaptive methods in terms of DOF and CPU-time.

\begin{figure}[H]
\centering
\hspace{-50mm}
\includegraphics[width=7.5cm]{nist/nist-4/conv_dof_aniso.png}\ \
\hspace{-10mm}
\includegraphics[width=7.5cm]{nist/nist-4/conv_cpu_aniso.png}
\hspace{-50mm}
\caption{DOF and CPU time convergence graphs.}
\label{fig:nist-4-conv}
\end{figure}

%%%%%%%%%%%%%%%%%%%%%%%%%%%%%%%%%%%%%%%%%%%%%%%%

\section{Benchmark NIST-5 "Battery"}
\label{sec:bench-5}

This is a heat conduction problem in a nonhomogeneous material.
It comes with an anisotropic solution with strong internal disruption
layers and singularities.
The solution has multiple point singularities in the interior at which
more than three different materials meet. These singularities are stronger than those
corresponding to reentrant corners. The equation solved is

\begin{equation} \label{heat-conduction}
-\frac{\partial }{\partial x}\left(p(x, y)\frac{\partial u}{\partial x}\right)
-\frac{\partial }{\partial y}\left(q(x, y)\frac{\partial u}{\partial y}\right) = f
\end{equation}
in the domain $\Omega = (0, 8.4) \times (0, 24)$. Boundary conditions are zero Neumann on left edge, Newton on the rest of the boundary.
The right-hand side $f$ are constant functions (different in respective materials).
The solution to NIST-5 is shown in Fig. \ref{fig:sln-nist05}.

\begin{figure}[H]
\centering
\vspace{-3mm}
\includegraphics[height=5cm]{nist/nist-5/solution.png}
\vspace{-3mm}
\caption{Solution to the NIST-5 benchmark problem.}
\label{fig:sln-nist05}
\end{figure}

Fig. \ref{fig:nist-5-hp-aniso} presents the final meshes corresponding to adaptive $h$-FEM with 
linear elements, adaptive $h$-FEM with quadratic elements, and adaptive $hp$-FEM. Different 
polynomial degrees of elements are represented by different colors. 

\begin{figure}[H]
\centering
\includegraphics[height=5cm]{nist/nist-5/mesh_h1_aniso.png}
\includegraphics[height=5cm]{nist/nist-5/mesh_h2_aniso.png}
\includegraphics[height=5cm]{nist/nist-5/mesh_hp_aniso.png}
\caption{
Left: Final mesh with 55577 DOF and relative error 9.57345e-02~\% for $h$-FEM with linear elements.
Middle: Final mesh with 12483 DOF and relative error 1.34925e-02~\% for $h$-FEM with quadratic elements. 
Right: Final mesh with 7450 DOF and relative error 1.46775e-02~\% for $hp$-FEM (adaptivity option HP\_ANISO\_H).}
\label{fig:nist-5-hp-aniso}
\end{figure}

Fig. \ref{fig:nist-5-conv} shows the convergence of the adaptive methods in terms of DOF and CPU-time.

\begin{figure}[H]
\centering
\hspace{-50mm}
\includegraphics[width=7.5cm]{nist/nist-5/conv_dof_aniso.png}\ \
\hspace{-10mm}
\includegraphics[width=7.5cm]{nist/nist-5/conv_cpu_aniso.png}
\hspace{-50mm}
\caption{DOF and CPU time convergence graphs.}
\label{fig:nist-5-conv}
\end{figure}

%%%%%%%%%%%%%%%%%%%%%%%%%%%%%%%%%%%%%%%%%%%%%%%%

\section{Benchmark NIST-6 "Boundary Layer"}
\label{sec:bench-6}

This example is a singularly perturbed problem with known exact solution that exhibits
a boundary layer along the right and top sides of the domain.
It is a convection-diffusion equation with first order terms.

\begin{equation} \label{boundary-layer}
-\epsilon \nabla^{2} u + 2\frac{\partial u}{\partial x} + \frac{\partial u}{\partial y} = f
\end{equation}
in the domain $\Omega = (-1, 1)^2$, equipped with Dirichlet boundary condition
given by the exact solution. The exact solution is
$u(x,y) = (1 - e^{-(1 - x) / \epsilon})(1 - e^{-(1 - y) / \epsilon})cos(\pi (x + y))$,
where $\epsilon$ determines the strength of the boundary layer.
The solution to NIST-6 containing a boundary layer
with $\epsilon = 10^{-1}$ is shown in Fig. \ref{fig:sln-nist06}.

\begin{figure}[H]
\centering
\vspace{-3mm}
\includegraphics[height=5cm]{nist/nist-6/solution.png}
\caption{Solution to the NIST-6 benchmark problem.}
\vspace{-3mm}
\label{fig:sln-nist06}
\end{figure}

Fig. \ref{fig:nist-6-hp-aniso} presents the final meshes corresponding to adaptive $h$-FEM with 
linear elements, adaptive $h$-FEM with quadratic elements, and adaptive $hp$-FEM. Different 
polynomial degrees of elements are represented by different colors. 

\begin{figure}[H]
\centering
%\vspace{-3mm}
\includegraphics[height=3.7cm]{nist/nist-6/mesh_h1_aniso.png}
\includegraphics[height=3.7cm]{nist/nist-6/mesh_h2_aniso.png}
\includegraphics[height=3.7cm]{nist/nist-6/mesh_hp_aniso.png}
\vspace{-3mm}
\caption{
Left: Final mesh with 55090 DOF and relative error 8.74769e-01~\% for $h$-FEM with linear elements.
Middle: Final mesh with 63145 DOF and relative error 1.46642e-02~\% for $h$-FEM with quadratic elements. 
Right: Final mesh with 591 DOF and relative error 6.23458e-04~\% for $hp$-FEM (adaptivity option HP\_ANISO\_H).}
\vspace{-2mm}
\label{fig:nist-6-hp-aniso}
\end{figure}

Fig. \ref{fig:nist-6-conv} shows the convergence of the adaptive methods in terms of DOF and CPU-time.

\begin{figure}[H]
\centering
\vspace{-2mm}
\hspace{-50mm}
\includegraphics[width=7.5cm]{nist/nist-6/conv_dof_aniso.png}\ \
\hspace{-10mm}
\includegraphics[width=7.5cm]{nist/nist-6/conv_cpu_aniso.png}
\hspace{-50mm}
\caption{DOF and CPU time convergence graphs.}
\vspace{-2mm}
\label{fig:nist-6-conv}
\end{figure}

%%%%%%%%%%%%%%%%%%%%%%%%%%%%%%%%%%%%%%%%%%%%%%%%

\section{Benchmark NIST-7 "Boundary Line Singularity"}
\label{sec:bench-7}

This is a singularity problem with the solution that is singular along the left part of the boundary.
The equation solved in this problem is the Poisson equation.

\begin{equation} \label{boundary-line-singularity}
-\Delta u = f
\end{equation}
in the domain $\Omega = (0, 1)^2$, equipped with Dirichlet boundary conditions
given by the exact solution. The exact solution is
$u(x,y) = x^{\alpha}$,
where $\alpha \geq 0.5$ determines the strength of the singularity.
The solution to NIST-7 with $\alpha = 0.6$ is shown in Fig. \ref{fig:sln-nist07}.

\begin{figure}[H]
\centering
%\vspace{-3mm}
\includegraphics[height=5cm]{nist/nist-7/solution.png}
\vspace{-3mm}
\caption{Solution to the NIST-7 benchmark problem.}
%\vspace{-2mm}
\label{fig:sln-nist07}
\end{figure}

Fig. \ref{fig:nist-7-hp-aniso} presents the final meshes corresponding to adaptive $h$-FEM with 
linear elements, adaptive $h$-FEM with quadratic elements, and adaptive $hp$-FEM. Different 
polynomial degrees of elements are represented by different colors. 

\begin{figure}[H]
\centering
\vspace{-5mm}
\includegraphics[height=3.7cm]{nist/nist-7/mesh_h1_aniso.png}
\includegraphics[height=3.7cm]{nist/nist-7/mesh_h2_aniso.png}
\includegraphics[height=3.7cm]{nist/nist-7/mesh_hp_aniso.png}
\vspace{-3mm}
\caption{
Left: Final mesh with 684 DOF and relative error 1.45724~\% for $h$-FEM with linear elements.
Middle: Final mesh with 267 DOF and relative error 1.49585~\% for $h$-FEM with quadratic elements. 
Right: Final mesh with 88 DOF and relative error 1.46348~\% for $hp$-FEM (adaptivity option HP\_ANISO\_H).}
\vspace{-3mm}
\label{fig:nist-7-hp-aniso}
\end{figure}

Fig. \ref{fig:nist-7-conv} shows the convergence of the adaptive methods in terms of DOF and CPU-time.

\begin{figure}[H]
\centering
\hspace{-50mm}
\includegraphics[width=7.5cm]{nist/nist-7/conv_dof_aniso.png}\ \
\hspace{-10mm}
\includegraphics[width=7.5cm]{nist/nist-7/conv_cpu_aniso.png}
\hspace{-50mm}
\caption{DOF and CPU time convergence graphs.}
\label{fig:nist-7-conv}
\vspace{-3mm}
\end{figure}

%%%%%%%%%%%%%%%%%%%%%%%%%%%%%%%%%%%%%%%%%%%%%%%%

\section{Benchmark NIST-8 "Oscillatory"}
\label{sec:bench-8}

This is a wave function that satisfies the Schr\"{o}dinger's equation model of two
interacting atoms, highly oscillatory near the origin.
The equation solved in this problem is the Helmholtz equation.

\begin{equation} \label{oscillatory}
-\nabla^{2} u - \frac{1}{(\alpha + r)^{4}} u = f
\end{equation}
in the domain $\Omega = (0, 1)^2$, equipped with Dirichlet boundary conditions
given by the exact solution. The exact solution is
$u(x,y) = sin(\frac{1}{\alpha + r})$,
where $r = \sqrt{x^{2} + y^{2}}$, $\alpha = 1 / N \pi$ determines the number of oscillations.
The solution to NIST-8 with $\alpha = 1 / 10 \pi$ is shown in Fig. \ref{fig:sln-nist08}.

\begin{figure}[H]
\centering
\vspace{-3mm}
\includegraphics[height=5cm]{nist/nist-8/solution.png}
\vspace{-3mm}
\caption{Solution to the NIST-8 benchmark problem.}
\vspace{-4mm}
\label{fig:sln-nist08}
\end{figure}


Fig. \ref{fig:nist-8-hp-aniso} presents the final meshes corresponding to adaptive $h$-FEM with 
linear elements, adaptive $h$-FEM with quadratic elements, and adaptive $hp$-FEM. Different 
polynomial degrees of elements are represented by different colors. 

\begin{figure}[H]
\centering
%\vspace{-5mm}
\includegraphics[height=3.7cm]{nist/nist-8/mesh_h1_aniso.png}
\includegraphics[height=3.7cm]{nist/nist-8/mesh_h2_aniso.png}
\includegraphics[height=3.7cm]{nist/nist-8/mesh_hp_aniso.png}
%\vspace{-3mm}
\caption{
Left: Final mesh with 55731 DOF and relative error 1.24198~\% for $h$-FEM with linear elements.
Middle: Final mesh with 3413 DOF and relative error 6.64833e-01~\% for $h$-FEM with quadratic elements. 
Right: Final mesh with 1160 DOF and relative error 9.09848e-02~\% for $hp$-FEM (adaptivity option HP\_ANISO\_H).}
%\vspace{-5mm}
\label{fig:nist-8-hp-aniso}
\end{figure}

Fig. \ref{fig:nist-8-conv} shows the convergence of the adaptive methods in terms of DOF and CPU-time.

\begin{figure}[H]
\centering
%\vspace{-4mm}
\hspace{-50mm}
\includegraphics[width=7.5cm]{nist/nist-8/conv_dof_aniso.png}\ \
\hspace{-10mm}
\includegraphics[width=7.5cm]{nist/nist-8/conv_cpu_aniso.png}
\hspace{-50mm}
\caption{DOF and CPU time convergence graphs.}
\vspace{-3mm}
\label{fig:nist-8-conv}
\end{figure}

%%%%%%%%%%%%%%%%%%%%%%%%%%%%%%%%%%%%%%%%%%%%%%%%

\section{Benchmark NIST-9 "Wave Front"}
\label{sec:bench-9}

This is a commonly used example for testing the performance of
adaptive refinement algorithms using a wave front and a singularity \cite{mitchell-1, mitchell-2}.
The solution has a sharp circular wave front in the interior of the
domain, with a singularity at the center of the circle.
The equation solved is the Poisson equation.

\begin{equation} \label{wave-front}
-\Delta u = f
\end{equation}
in the domain $\Omega = (0, 1)^2$, equipped with Dirichlet boundary conditions
given by the exact solution. The exact solution is
$u(x, y) = tan^{-1}(\alpha (r - r_{0}))$,
where $r = \sqrt{(x - x_{loc})^{2} + (y - y_{loc})^{2}}$.
Here $(x_{loc}, y_{loc})$ is the center of the circular wave front,
$r_{0}$ is the distance from the wave front to the center of the circle,
and $\alpha$ gives the steepness of the wave front.
The solution to NIST-9 with $\alpha = 50$, $(x_{loc}, y_{loc}) = (0.5, 0.5)$,
$r_{0} = 0.25$ is shown in Fig. \ref{fig:sln-nist09}.

\begin{figure}[H]
\centering
\vspace{-3mm}
\includegraphics[height=5cm]{nist/nist-9/solution.png}
\vspace{-3mm}
\caption{Solution to the NIST-9 benchmark problem.}
\vspace{-3mm}
\label{fig:sln-nist09}
\end{figure}

Fig. \ref{fig:nist-9-hp-aniso} presents the final meshes corresponding to adaptive $h$-FEM with 
linear elements, adaptive $h$-FEM with quadratic elements, and adaptive $hp$-FEM. Different 
polynomial degrees of elements are represented by different colors. 

\begin{figure}[H]
\centering
%\vspace{-5mm}
\includegraphics[height=3.7cm]{nist/nist-9/mesh_h1_aniso.png}
\includegraphics[height=3.7cm]{nist/nist-9/mesh_h2_aniso.png}
\includegraphics[height=3.7cm]{nist/nist-9/mesh_hp_aniso.png}
%\vspace{-5mm}
\caption{
Left: Final mesh with 46093 DOF and relative error 1.23973~\% for $h$-FEM with linear elements.
Middle: Final mesh with 4489 DOF and relative error 7.19928e-01~\% for $h$-FEM with quadratic elements. 
Right: Final mesh with 1465 DOF and relative error 8.67667e-01~\% for $hp$-FEM (adaptivity option HP\_ANISO\_H).}
%\vspace{-5mm}
\label{fig:nist-9-hp-aniso}
\end{figure}

Fig. \ref{fig:nist-9-conv} shows the convergence of the adaptive methods in terms of DOF and CPU-time.

\begin{figure}[H]
\centering
\vspace{-3mm}
\hspace{-50mm}
\includegraphics[width=7.5cm]{nist/nist-9/conv_dof_aniso.png}\ \
\hspace{-10mm}
\includegraphics[width=7.5cm]{nist/nist-9/conv_cpu_aniso.png}
\hspace{-50mm}
\caption{DOF and CPU time convergence graphs.}
%\vspace{-4mm}
\label{fig:nist-9-conv}
\end{figure}

%%%%%%%%%%%%%%%%%%%%%%%%%%%%%%%%%%%%%%%%%%%%%%%%

\section{Benchmark NIST-10 "Interior Line Singularity"}
\label{sec:bench-10}

This is another example with anisotropic solution that is suitable for testing
anisotropic element refinements. The equation solved is the Poisson equation.
%\vspace{-2mm}
\begin{equation} \label{interior}
-\Delta u = f
\end{equation}
in the domain $\Omega = (-1, 1)^2$, equipped with a zero
Neumann boundary condition on left edge, Dirichlet boundary
conditions given by the exact solution on the rest of the boundary.
The exact solution is
$u(x,y) = \cos(Ky)\ (x \le 0)$ and $u(x,y) = \cos(Ky) + x^{\alpha}\ (x > 0)$,
where $K$ and $\alpha$ are constants.
The solution to NIST-10 containing a line singularity with $K = \pi/2$ and
$\alpha = 2.01$ is shown in Fig. \ref{fig:sln-nist10}.

\begin{figure}[H]
\centering
\vspace{-6mm}
\includegraphics[height=5cm]{nist/nist-10/solution.png}
\vspace{-3mm}
\caption{Solution to the NIST-10 benchmark problem.}
\vspace{-3mm}
\label{fig:sln-nist10}
\end{figure}

Fig. \ref{fig:nist-10-hp-aniso} presents the final meshes corresponding to adaptive $h$-FEM with 
linear elements, adaptive $h$-FEM with quadratic elements, and adaptive $hp$-FEM. Different 
polynomial degrees of elements are represented by different colors. 

\begin{figure}[H]
\centering
\vspace{-3mm}
\includegraphics[height=3.7cm]{nist/nist-10/mesh_h1_aniso.png}
\includegraphics[height=3.7cm]{nist/nist-10/mesh_h2_aniso.png}
\includegraphics[height=3.7cm]{nist/nist-10/mesh_hp_aniso.png}
\vspace{-3mm}
\caption{
Left: Final mesh with 27999 DOF and relative error 2.87268e-01~\% for $h$-FEM with linear elements.
Middle: Final mesh with 60144 DOF and relative error 2.59816e-04~\% for $h$-FEM with quadratic elements. 
Right: Final mesh with 381 DOF and relative error 8.68994e-05~\% for $hp$-FEM (adaptivity option HP\_ANISO\_H).}
\vspace{-3mm}
\label{fig:nist-10-hp-aniso}
\end{figure}


\begin{figure}[H]
\centering
\vspace{-3mm}
\hspace{-50mm}
\includegraphics[width=7.5cm]{nist/nist-10/conv_dof_aniso.png}\ \
\hspace{-10mm}
\includegraphics[width=7.5cm]{nist/nist-10/conv_cpu_aniso.png}
\hspace{-50mm}
\vspace{-3mm}
\caption{DOF and CPU time convergence graphs.}
\label{fig:nist-10-conv}
\end{figure}

Fig. \ref{fig:nist-10-conv} shows the convergence of the adaptive methods in terms of DOF and CPU-time.

%%%%%%%%%%%%%%%%%%%%%%%%%%%%%%%%%%%%%%%%%%%%%%%%

\section{Benchmark NIST-11 "Intersecting Interfaces"}
\label{sec:bench-11}

This is a Poisson problem with intersecting interfaces,
dividing the plane into four regions.
The solution to this problem contains a severe
singularity that poses a challenge to adaptive methods.
The equation solved is given by

\begin{equation} \label{intersecting}
-\nabla \cdot (a(x,y) \nabla u) = 0
\end{equation}
where the parameter $a$ is piecewise-constant,
$a(x,y) = 161.4476387975881$ in the first and third quadrants,
and $a(x,y) = 1$ in the remaining two quadrants.
The domain of this problem is $\Omega = (-1, 1)^2$, equipped with
Dirichlet boundary conditions given by the exact solution.
The exact solution is
$u(x,y) = r^{a_1} \mu (\theta)$,
where $a_1$ and $\mu (\theta)$ is given in \cite{mitchell-1}.
The solution to NIST-11 is shown in Fig. \ref{fig:sln-nist11}.

\begin{figure}[H]
\centering
%\vspace{-5mm}
\includegraphics[height=4.7cm]{nist/nist-11/solution.png}
%\vspace{-2mm}
\caption{Solution to the NIST-11 benchmark problem.}
\vspace{-3mm}
\label{fig:sln-nist11}
\end{figure}

Fig. \ref{fig:nist-11-hp-aniso} presents the final meshes corresponding to adaptive $h$-FEM with 
linear elements, adaptive $h$-FEM with quadratic elements, and adaptive $hp$-FEM. Different 
polynomial degrees of elements are represented by different colors. 

\begin{figure}[H]
\centering
%\vspace{-2mm}
\includegraphics[height=3.7cm]{nist/nist-11/mesh_h1_aniso.png}
\includegraphics[height=3.7cm]{nist/nist-11/mesh_h2_aniso.png}
\includegraphics[height=3.7cm]{nist/nist-11/mesh_hp_aniso.png}
\vspace{-3mm}
\caption{
Left: Final mesh with 46905 DOF and relative error 1.25659~\% for $h$-FEM with linear elements.
Middle: Final mesh with 7777 DOF and relative error 4.73604e-01~\% for $h$-FEM with quadratic elements. 
Right: Final mesh with 3459 DOF and relative error 4.7087e-01~\% for $hp$-FEM (adaptivity option HP\_ANISO\_H).}
\vspace{-2mm}
\label{fig:nist-11-hp-aniso}
\end{figure}

Fig. \ref{fig:nist-11-conv} shows the convergence of the adaptive methods in terms of DOF and CPU-time.

\begin{figure}[H]
\centering
\hspace{-50mm}
\includegraphics[width=7.5cm]{nist/nist-11/conv_dof_aniso.png}\ \
\hspace{-10mm}
\includegraphics[width=7.5cm]{nist/nist-11/conv_cpu_aniso.png}
\hspace{-50mm}
\caption{DOF and CPU time convergence graphs.}
\label{fig:nist-11-conv}
\end{figure}

%%%%%%%%%%%%%%%%%%%%%%%%%%%%%%%%%%%%%%%%%%%%%%%%

\section{Benchmark NIST-12 "Multiple Difficulties"}
\label{sec:bench-12}

This problem combines four aspects of benchmarks
seen in previous sections (nist-2, nist-4, nist-6 and nist-9) into one problem.
The wave front intersects the boundary
layer and corner singularity, and the peak is centered on the wave front.
The equation solved is the Poisson equation.
\vspace{-3mm}
\begin{equation} \label{multiple}
-\Delta u = f
\end{equation}
in the L-shaped domain, equipped with Dirichlet boundary conditions
given by the exact solution.
The exact solution is

\[
u(x,y) =  r^{\alpha_{C} }\sin(\alpha_{C} \theta)
+ e^{-\alpha_{P} ((x - x_{P})^{2} + (y - y_{P})^{2})}
+ tan^{-1}(\alpha_{W} (r_{W} - r_{0})) \\
+ e^{-(1 - y) / \epsilon}
\]
where $\alpha_C = \pi / \omega_C$, $r = \sqrt{x^2+y^2}$
and $\theta = tan^{-1}(y/x)$, here $\omega_C$ determines
the angle of the re-entrant corner.
$(x_{P}, y_{P})$ is the location of the peak, $\alpha$
determines the strength of the peak. Furthermore
$r_{W} = \sqrt{(x - x_{W})^{2} + (y - y_{W})^{2}}$,
where $(x_{W}, y_{W})$ is the center of the circular wave front,
$r_{0}$ is the distance from the wave front to the
center of the circle, and $\alpha_W$ gives
the steepness of the wave front. The parameter $\epsilon$ determines the
strength of the boundary layer, the boundary layer was placed on $y = -1$.
The solution to NIST-12 with $\omega_C = 3 \pi /2$,
$(x_{W}, y_{W}) = (0, -3/4)$, $r_{0} = 3/4$, $\alpha_{W} = 200$,
$(x_{P}, y_{P}) = (\sqrt{5} / 4, -1/4)$,
$\epsilon = 1/100$ is shown in Fig. \ref{fig:sln-nist12}.

\begin{figure}[H]
\centering
%\vspace{-5mm}
\includegraphics[height=5cm]{nist/nist-12/solution.png}
\vspace{-3mm}
\caption{Solution to the NIST-12 benchmark problem.}
\vspace{-4mm}
\label{fig:sln-nist12}
\end{figure}

Fig. \ref{fig:nist-12-hp-aniso} presents the final meshes corresponding to adaptive $h$-FEM with 
linear elements, adaptive $h$-FEM with quadratic elements, and adaptive $hp$-FEM. Different 
polynomial degrees of elements are represented by different colors. 

\begin{figure}[H]
\centering
\vspace{-3mm}
\includegraphics[height=3.7cm]{nist/nist-12/mesh_h1_aniso.png}
\includegraphics[height=3.7cm]{nist/nist-12/mesh_h2_aniso.png}
\includegraphics[height=3.7cm]{nist/nist-12/mesh_hp_aniso.png}
\vspace{-3mm}
\caption{
Left: Final mesh with 45533 DOF and relative error 2.05698~\% for $h$-FEM with linear elements.
Middle: Final mesh with 10495 DOF and relative error 9.04458e-01~\% for $h$-FEM with quadratic elements. 
Right: Final mesh with 4438 DOF and relative error 9.85118e-01~\% for $hp$-FEM (adaptivity option HP\_ANISO\_H).}
\vspace{-3mm}
\label{fig:nist-12-hp-aniso}
\end{figure}

Fig. \ref{fig:nist-12-conv} shows the convergence of the adaptive methods in terms of DOF and CPU-time.

\begin{figure}[H]
\centering
\hspace{-50mm}
\includegraphics[width=7.5cm]{nist/nist-12/conv_dof_aniso.png}\ \
\hspace{-10mm}
\includegraphics[width=7.5cm]{nist/nist-12/conv_cpu_aniso.png}
\hspace{-50mm}
\caption{DOF and CPU time convergence graphs.}
\label{fig:nist-12-conv}
\end{figure}
\vspace{-8mm}
%%%%%%%%%%%%%%%%%%%%%%%%%%%%%%%%%%%%%%%%%%%%%%%%%%%%%%%

\section{Conclusion and Outlook}
\label{sec:conclusion}

A challenging set of benchmarks aimed at testing adaptive Finite Element Method implementations in terms of handling diverse problems, as well as the solutions of these problems obtained by Hermes, have been presented in this paper.

Hermes also allowed for the comparison of anisotropic hp-FEM to low order (anisotropic) h-FEM.

The numerical results are given in such a way to make it possible to compare them to results obtained with another implementation of adaptive Finite Element Method.

\section{Acknowledgment}

This work was supported by Subcontract No. 00089911 of Battelle
Energy Alliance (DOE intermediary) as well as by the
Grant No. IAA100760702 of the Grant Agency of the Academy
of Sciences of the Czech Republic. The first autor was partly
supported by the National Natural Science Foundation
of China under Projects No. 41074099.

\section{Appendix 1 - Automatic hp-adaptivity with the open source library Hermes}
Hermes is a free C++/Python library for rapid development of adaptive hp-FEM and hp-DG solvers for partial differential equations (PDE) and multiphysics PDE systems. Hermes has eight different adaptivity options P\_ISO, P\_ANISO, H\_ISO, H\_ANISO, HP\_ISO, HP\_ANISO\_P, HP\_ANISO\_H, HP\_ANISO. H here stands for h-adaptivity, P for p-adaptivity, ISO for only isotropic refinements, ANISO for anisotropic (ANISO\_H means anisotropic refinements made possible only in space, ANISO\_P only in the degree of polynomials). In this order, usually P\_ISO yields the worst results and HP\_ANISO the best. However, even P\_ISO can be very efficient with a good starting mesh. 

You can try the benchmarks for yourself, the webpage address where you can download Hermes, and link to installation instructions are below. The benchmarks are located in\\
/hermes2d/benchmarks/nist-01, ..., nist-12.
\\ \\
Main Hermes webpage: http://hpfem.org/hermes/\\
Installation instructions: \small{http://hpfem.org/hermes/doc/index.html\#installation}

%\noindent

%% Authors are advised to submit their bibtex database files. They are
%% requested to list a bibtex style file in the manuscript if they do
%% not want to use elsarticle-num.bst.

%% References without bibTeX database:

% \begin{thebibliography}{00}	

%% \bibitem must have the following form:
%%   \bibitem{key}...
%%

% \bibitem{}

% \end{thebibliography}

\begin{thebibliography}{[KLR73]}

\bibitem{mitchell-1}
W. Mitchell: A Collection of 2D Elliptic Problems for
Testing Adaptive Algorithms, NISTIR 7668, 2010 (available online).

\vspace{-2mm}

\bibitem{mitchell-2}
W. Mitchell: A Survey of hp-Adaptive Strategies for Elliptic Partial Differential Equations,
Annals of the European Academy of Sciences, to appear (available online).

\vspace{-2mm}

%\bibitem{demkowicz-1}
%L. Demkowicz: One and Two Dimensional Elliptic and Maxwell Problems,
%Chapman \& Hall \/ CRC Press, Taylor \& Francis, 2006.


\bibitem{label2}
P. Solin, D. Andrs, J. Cerveny, M. Simko:
PDE-Independent Adaptive $hp$-FEM Based on Hierarchic Extension of Finite Element Spaces.
J. Comput. Appl. Math. 233 (2010) 3086-3094.

\vspace{-2mm}

\bibitem{thermoel}
P. Solin, J. Cerveny, L. Dubcova, D. Andrs:
Monolithic Discretization of Linear Thermoelasticity Problems
via Adaptive Multimesh $hp$-FEM, J. Comput. Appl. Math 234 (2010) 2350 - 2357.

\vspace{-2mm}

\bibitem{sosedo}
P. Solin. K. Segeth, I. Dolezel: Higher-Order Finite Element Methods, Chapman \& Hall
/ CRC Press, Boca Raton, 2003.
\end{thebibliography}

%\newpage

\end{document}

%%
%% End of file `elsarticle-template-num.tex'.
