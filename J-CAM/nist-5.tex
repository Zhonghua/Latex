\section{Benchmark NIST-5 "Battery"}
\label{sec:bench-5}

This is a heat conduction problem in a nonhomogeneous material. It comes with an anisotropic solution and
multiple singularities. The solution has multiple point singularities in the interior of the domain.
The equation solved is given by
\begin{equation} \label{heat-conduction}
-\frac{\partial }{\partial x}\left(p(x, y)\frac{\partial u}{\partial x}\right)
-\frac{\partial }{\partial y}\left(q(x, y)\frac{\partial u}{\partial y}\right) = f,
\end{equation}
in the domain $\Omega = (0, 8.4) \times (0, 24)$, equipped with zero Neumann boundary condition on the left edge, Natural boundary conditions on the rest of the boundary:

\begin{equation}
p(x, y)\frac{\partial u}{\partial x}\nu_1 + q(x, y)\frac{\partial u}{\partial y}\nu_2 = g_{left}(x, y) \ \mbox{on} \  \Gamma_{left}
\end{equation}

\begin{equation}
p(x, y)\frac{\partial u}{\partial x}\nu_1 + q(x, y)\frac{\partial u}{\partial y}\nu_2 + c(x, y)u = g_{right}(x, y) \ \mbox{on} \ \Gamma_{right}
\end{equation}

\begin{equation}
p(x, y)\frac{\partial u}{\partial x}\nu_1 + q(x, y)\frac{\partial u}{\partial y}\nu_2 + c(x, y)u = g_{top}(x, y) \ \mbox{on} \ \Gamma_{top}
\end{equation}

\begin{equation}
p(x, y)\frac{\partial u}{\partial x}\nu_1 + q(x, y)\frac{\partial u}{\partial y}\nu_2 + c(x, y)u = g_{bottom}(x, y) \ \mbox{on} \ \Gamma_{bottom},
\end{equation}
where $p(x, y)$, $q(x, y)$, $c(x, y)$, $g(x, y)$, and the right hand side $f$ are constant functions (different in respective materials).

The solution of NIST-5 is shown in Fig. \ref{fig:sln-nist05}.
\begin{figure}[!ht]
\centering
\includegraphics[height=5cm]{nist/nist-5/solution.png}
\caption{The solution to NIST-5 benchmark problem.}
\label{fig:sln-nist05}
\end{figure}

The goal of the benchmark is to reach a relative error below
$10^{-1}$~\% in the $H^1$-norm with as few DOF
as possible.
We begin with adaptive $hp$-FEM with possibly anisotropic refinements.
The initial mesh is shown in Fig. \ref{fig:nist-5-hp-aniso} (left).
In a few adaptivity steps, the polynomial degree of this domain is increased
anisotropically.
The resulting mesh with 2711 DOF is shown in Fig. \ref{fig:nist-5-hp-aniso} (right).

\begin{figure}[!ht]
\centering
\includegraphics[height=5cm]{nist/nist-5/mesh_hp_aniso_init.png}\ \
\includegraphics[height=5cm]{nist/nist-5/mesh_hp_aniso.png}
\vspace{-2mm}
\caption{Initial mesh (left) and final mesh (right) for $hp$-FEM with anisotropic refinements.}
\label{fig:nist-5-hp-aniso}
\end{figure}

The final relative error estimate in $H^1$-norm was 9.17542e-02 \%,
and it was identical to the exact error in all printed digits.
We also solved this benchmark with adaptive $h$-FEM
with linear (left) and quadratic (right)
elements, with anisotropic refinements enabled.
Final meshes for the $h$-FEM computations are shown
in Fig. \ref{fig:nist-5-h-aniso}.

\begin{figure}[!ht]
\centering
\includegraphics[height=5cm]{nist/nist-5/mesh_h1_aniso.png}\ \
\includegraphics[height=5cm]{nist/nist-5/mesh_h2_aniso.png}
\vspace{-2mm}
\caption{Final mesh for $h$-FEM anisotropic refinements with linear and quadratic elements.}
\label{fig:nist-5-h-aniso}
\end{figure}

Figs. \ref{fig:nist-5-conv} compare all
three approaches to automatic adaptivity from the point
of view of DOF and CPU convergence.

\begin{figure}[!ht]
\centering
\includegraphics[height=5cm]{nist/nist-5/conv_dof_aniso.png}\ \
\includegraphics[height=5cm]{nist/nist-5/conv_cpu_aniso.png}
\caption{DOF and CPU time convergence graphs.}
\label{fig:nist-5-conv}
\end{figure}

