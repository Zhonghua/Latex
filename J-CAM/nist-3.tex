\section{Benchmark NIST-3 "Linear Elasticity"}
\label{sec:bench-3}

Linear elasticity is used extensively in structural analysis
and engineering. Linear elasticity is a simplification
of the general nonlinear equations of elasticity and is the mathematical
study of how solid objects deform and become internally
stressed due to prescribed loading conditions.
In this benchmark, we present a standard system of two
coupled equations with mixed derivative for linear elasticity
in the coupling term. This example employs the adaptive multimesh $hp$-FEM
to solve the equations.

\begin{equation}\label{crack}
\left\{
\begin{array}{l}
\displaystyle
-E \frac{1-\nu^2}{1-2\nu} \frac{\partial^{2} u}{\partial x^{2}} - E\frac{1-\nu^2}{2-2 \nu} \frac{\partial^{2} u}{\partial y^{2}}
-E \frac{1-\nu^2}{(1-2\nu)(2-2\nu)} \frac{\partial^{2} v}{\partial x \partial y} = F_{x} \\
\displaystyle
-E \frac{1-\nu^2}{2-2\nu} \frac{\partial^{2} v}{\partial x^{2}} - E\frac{1-\nu^2}{1-2\nu} \frac{\partial^{2} v}{\partial y^{2}}
-E \frac{1-\nu^2}{(1-2\nu)(2-2\nu)} \frac{\partial^{2} u}{\partial x \partial y} = F_{y}
\end{array}
\right.
\end{equation}
where $F_{x} = F_{y} = 0$, $u$ and $v$ are the
$x$ and $y$ displacements, $E$ is Young's Modulus,
and $\nu$ is Poisson's ratio.

The domain in the example is $\Omega = (-1, 1)^2$ with a slit,
equipped with Dirichlet boundary conditions given by the
exact solution. The exact solution of (\ref{crack}) in polar coordinates is
$u(x, y) = \frac{1}{2G} r^{\lambda}[(k - Q(\lambda + 1))cos(\lambda \theta) - \lambda cos((\lambda - 2) \theta)]$ and
$v(x, y) = \frac{1}{2G} r^{\lambda}[(k + Q(\lambda + 1))sin(\lambda \theta) + \lambda sin((\lambda - 2) \theta)]$,
where $\lambda = 0.5444837367825$, $Q = 0.5430755788367$,
$k = 3 - 4 \nu$ and $G = E / (2(1 + \nu))$.
The solution of NIST-3 is shown in Fig. \ref{fig:sln-nist03}.

\begin{figure}[!ht]
\centering
\includegraphics[height=4cm]{nist/nist-3/solution-u.png}\ \
\includegraphics[height=4cm]{nist/nist-3/solution-v.png}
\caption{The $u$ (left) and $v$ (right) component to NIST-3 benchmark problem.}
\label{fig:sln-nist03}
\end{figure}

\begin{figure}[!ht]
\centering
\includegraphics[height=3.7cm]{nist/nist-3/mesh_u_h1_aniso.png}
\includegraphics[height=3.7cm]{nist/nist-3/mesh_u_h2_aniso.png}
\includegraphics[height=3.7cm]{nist/nist-3/mesh_u_hp_anisoh.png}\ \
\includegraphics[height=3.7cm]{nist/nist-3/mesh_v_h1_aniso.png}
\includegraphics[height=3.7cm]{nist/nist-3/mesh_v_h2_aniso.png}
\includegraphics[height=3.7cm]{nist/nist-3/mesh_v_hp_anisoh.png}
\caption{
Final mesh (left) with 39779 DOF and the resulting
relative error estimate in $H^1$-norm of 3.84929e-01 \% for $h$-FEM with linear elements.
Final mesh (middle) with 9330 DOF and the resulting
relative error estimate in $H^1$-norm of 9.56383e-02 \% for $h$-FEM with quadratic elements.
Final mesh (right) with 3897 DOF and the resulting
relative error estimate in $H^1$-norm of 8.05635e-02 \% for $hp$-FEM with anisotropic refinements.}
\label{fig:nist-3-hp-aniso}
\end{figure}

%The final relative error estimate in $H^1$-norm was 8.05635e-02 \%,
%and it was identical to the exact error in all printed digits.
%We also solved this benchmark with adaptive $h$-FEM
%with linear and quadratic elements, with anisotropic refinements enabled.
%Final meshes for the $h$-FEM computations are shown
%in Fig. \ref{fig:nist-3-h1-aniso} and Fig. \ref{fig:nist-3-h2-aniso}.

%\begin{figure}[!ht]
%\centering
%\includegraphics[height=5cm]{nist/nist-3/mesh_u_h1_aniso.png}\ \
%\includegraphics[height=5cm]{nist/nist-3/mesh_v_h1_aniso.png}
%\caption{Final meshs of $u$ (left) and $v$ (right) component for $h$-FEM anisotropic refinements with linear elements.}
%\label{fig:nist-3-h1-aniso}
%\end{figure}
%
%\begin{figure}[!ht]
%\centering
%\includegraphics[height=5cm]{nist/nist-3/mesh_u_h2_aniso.png}\ \
%\includegraphics[height=5cm]{nist/nist-3/mesh_v_h2_aniso.png}
%\caption{Final meshs of $u$ (left) and $v$ (right) component for $h$-FEM anisotropic refinements with quadratic elements.}
%\label{fig:nist-3-h2-aniso}
%\end{figure}

Figs. \ref{fig:nist-3-conv} compare all
three approaches to automatic adaptivity from the point
of view of DOF and CPU convergence.

\begin{figure}[!ht]
\centering
\includegraphics[height=5cm]{nist/nist-3/conv_dof_aniso.png}\ \
\includegraphics[height=5cm]{nist/nist-3/conv_cpu_aniso.png}
\caption{DOF and CPU time convergence graphs.}
\label{fig:nist-3-conv}
\end{figure}

